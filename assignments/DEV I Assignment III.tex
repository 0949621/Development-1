\documentclass[11pt, a4paper]{article}
\usepackage{opgave}
\usepackage{fancyStyle}

\renewcommand{\course}{DEV01-1}
\renewcommand{\titel}{Assignment 3}
\renewcommand{\auteur}{Youri Tjang}
\renewcommand{\auteurlang}{\auteur\\ \href{mailto:y.s.tjang@hr.nl}{\texttt{y.s.tjang@hr.nl}}}

\setboolean{isTentamen}{false}
\setboolean{showAntwoord}{false}

\usepackage{graphicx}

\begin{document}
\setcounter{opgave}{1}


\opgave{}{1}
\subopgave{Write a program that allows any student to reach the goal.}{}{1}
\subopgave{Also use the memory model (program counter, variables, etc) shown in the slides. And write a complete run of your program.}{}{1}

\begin{itemize}
    \item No whiles and if’s
    \item You can use distances, ages, furniture, etc. Everything within the room.
    \item Between the student and the door there are no objects
    \item Make the assignment on your own
    \item Only use pen and paper
\end{itemize}

\opgave{}{1}
The same as exercise 1, but now include objects between the student and the door. You are allowed to use if’s but not whiles.

\opgave{}{1}
\subopgave{As exercise 2, but now include at least 1 while.}{}{1}
\subopgave{Discuss the differences between: is it shorter, is it more readable, are the amount of variables the same, etc?}{}{1}

\einde
\end{document}