\section*{Exam sample 2}
What follows is a concrete example of the exam.


\paragraph{Question I: formal rules} \ \\

\textit{You start at point (-10,20). Take 5 steps of magnitude (5,-5). Then take steps iof magnitude (0, 10) until you are above the line (0, 10). Where do you end up?}

\ \\ 

\textbf{Answer:} \textit{The trajectory is:}

\begin{lstlisting}
(-10,20) +
          \
           \   + (15,15)
            \  |
             \ |
              \|
        (15,-5)+



\end{lstlisting}

\ \\ 

\textbf{Points:} \textit{25\%.}

\ \\ 
\ \\ 

\paragraph{Question II: program state} \ \\ 

\textit{Fill-in the program state with the values that the variables assume while running the sample below.}

\begin{lstlisting}
count = 1
for i in range(1, 5):
    count *= i
\end{lstlisting}

\ \\ 

\textbf{Answer:} \textit{The variable allocations are:}
\begin{lstlisting}
count: 	 1	 2	 6	 24
i:	 1	 2	 3	 4
\end{lstlisting}

\ \\

\textbf{Points:} \textit{25\%.}

\ \\ 

\textbf{Grading:} \textit{Full points if all values are correctly listed in the right order. Half points if at least half of values are listed in the right order. Zero points otherwise.}

\ \\ 

\textbf{Associated learning goals:} \texttt{CMC}.

\ \\ 

\paragraph{Question III: variables, expressions, and data types}

\ \\ 

\textit{What is the value and the type of all variables after execution of the following code?}
\begin{lstlisting}
h = input("Wat is je naam?")
j = "Hello {}".format(h)
k = 10 / 3
l = k <= 3 or True
i = "Hello" + 1
\end{lstlisting}

\ \\ 

\textbf{Answer:} \textit{The value and type of all variables after execution is:}

\begin{tabular}{| l | c | c | }
\hline
\textbf{Variable} & \textbf{Value} & \textbf{Type} \\
\hline
h & 'Youri' & string \\
\hline
j & 'Hello Youri' & str \\
\hline
k & 3.333333 & float \\
\hline
l & True & boolean \\
\hline
i & Error, omdat je geen plus kan doen met een string en een int & error \\
\hline
\end{tabular}

\ \\ 

\textbf{Points:} \textit{25\%.}

\ \\ 

\textbf{Grading:} \textit{All values and types are correct: full-points. At least half the values and at least half the types are correct: half points. Zero points otherwise.}

\ \\ 

\textbf{Associated learning goals:} \texttt{VAR}, \texttt{EXPR}.

\ \\ 

\paragraph{Question IV: control flow}

\ \\ 

\textbf{General shape of the question:} \textit{What is the value of all variables after execution of the following code?}

\ \\ 

\textbf{Concrete example of question:} \textit{Draw what is printed on the screen after execution of the following code?}

\begin{lstlisting}
output = ""
for i in range(0, 4):
    for j in range(0, 4):
        if (i + j) % 2 == 0:
            output += "0"
        else:
            output += "="
    output += "\n"
print(output)
\end{lstlisting}

\ \\ 

\textbf{Concrete example of answer:} \textit{The screen looks like:}
\begin{lstlisting}

0=0=
=0=0
0=0=
=0=0

\end{lstlisting}

\ \\ 

\textbf{Points:} \textit{25\%.}

\ \\ 

\textbf{Grading:} \textit{All values are correct: full-points. At least half the values are correct: half points. Zero points otherwise.}

\ \\ 

\textbf{Associated learning goals:} \texttt{COND}, \texttt{LOOP}.

\ \\ 
