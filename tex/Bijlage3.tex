\section*{Bijlage 3: Studielast normering in ects}

		Dit hoofdstuk bevat de beschrijving van de procedures om voor beoordeling in aanmerking te komen.\\
		Bijvoorbeeld voldoende aanwezigheid, 80\% van de opdrachten hebben ingeleverd, presentaties hebben verricht etc.\\

		Verder wordt zo gedetailleerd mogelijk beschreven hoe er tot een cijfer wordt gekomen en welke rollen er door docenten en ander betrokkenen hierbij vervuld worden. \\

		Geef een verantwoording van de toets; wat wordt getoetst, waarom is voor deze vorm gekozen.\\

		Vul een toetsmatrijs in voor de toets (zie bijlage).\\

		Beschrijf ook duidelijk de \textbf{herkansingsmogelijkheden}. \\

		Neem in geval van een schriftelijk tentamen een voorbeeldtoets op als bijlage.\\
		Geef daarbij per deelvraag het aantal te verdienen punten aan. \\

		Bij een schriftelijk rapport. Geef de beoordelingscriteria aan met daarbij de mogelijke score en de onderlinge weging. \\

		Toetsduur: \\

		Hoe en wanneer krijgt de student feedback?\\