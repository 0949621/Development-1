\documentclass{beamer}
\usetheme[hideothersubsections]{HRTheme}
\usepackage{beamerthemeHRTheme}
\usepackage{graphicx}
\usepackage[space]{grffile}
\usepackage{listings}
\lstset{language=C,
basicstyle=\ttfamily\footnotesize,
mathescape=true,
breaklines=true}
\usepackage[utf8]{inputenc}

\title{Concrete model of computation}

\author{Development Team}

\institute{Hogeschool Rotterdam \\ 
Rotterdam, Netherlands}

\date{}

\begin{document}
\maketitle

\SlideSection{Introduction}
\SlideSubSection{Lecture topics}
\begin{slide}{
\item We discuss a formal way to define computation
\item We discuss the fundamental elements of a concrete computer
\item We bridge what we have seen in the previous lecture with concrete descriptions
}\end{slide}

\SlideSubSection{Semantics}
\begin{slide}{
\item Any language has \textbf{semantics}
\item \textbf{Semantics} describe the \textit{meaning} of sentences in the language
\item Programming languages have \textbf{formal semantics}
\item \textbf{Formal semantics} are expressed in a very logical, unambiguous format
}\end{slide}

\SlideSubSection{Semantics of \texttt{stdNt}}
\begin{frame}[fragile]
Consider this program from the previous lecture:

\begin{lstlisting}[frame=shadowbox,numbers=left]
take 3 steps forward 
sit on the chair 
turn left 
slide 3 steps forward
\end{lstlisting}

What do you implicitly assume by performing each of the instructions? \textbf{Try to guess and discuss!}
\end{frame}

\begin{slide}{
\item We start with a \textit{current instruction} and a \textit{student state}:
\begin{itemize}
\item The current instruction (often called \textit{instruction pointer} (\texttt{IP}) or \textit{program counter} (\texttt{PC})) is just the index of the current instruction;
\item the student state (usually just called \textit{state}, or \texttt{S}, or $\sigma$) is whatever relevant attributes we track about the student (for example, his position and orientation in the room and whether or not he is sitting).
\end{itemize}
\item Each instruction changes the \texttt{PC} and the \texttt{S}.
}\end{slide}

\begin{frame}[fragile]
\texttt{\center
\begin{tabular}{| l | l | l | l |}
\hline
PC & S.Pose & S.Orientation & S.Position \\
\hline
1 & Standing & Forward & (0,0) \\
\hline
\end{tabular}
} \ \\ \ \\ \ \\

\begin{lstlisting}[frame=shadowbox,numbers=left]
take 3 steps forward 
sit on the chair 
turn left 
slide 3 steps forward
\end{lstlisting}

what changes while running the current instruction? \textbf{Try to guess and discuss!}
\end{frame}

\begin{frame}[fragile]
\texttt{\center
\begin{tabular}{| l | l | l | l |}
\hline
PC & S.Pose & S.Orientation & S.Position \\
\hline
2 & Standing & Forward & (0,3) \\
\hline
\end{tabular}
} \ \\ \ \\ \ \\

\begin{lstlisting}[frame=shadowbox,numbers=left]
take 3 steps forward 
sit on the chair 
turn left 
slide 3 steps forward
\end{lstlisting}

what changes while running the current instruction? \textbf{Try to guess and discuss!}
\end{frame}

\begin{frame}[fragile]
\texttt{\center
\begin{tabular}{| l | l | l | l |}
\hline
PC & S.Pose & S.Orientation & S.Position \\
\hline
3 & Sitting & Forward & (0,3) \\
\hline
\end{tabular}
} \ \\ \ \\ \ \\

\begin{lstlisting}[frame=shadowbox,numbers=left]
take 3 steps forward 
sit on the chair 
turn left 
slide 3 steps forward
\end{lstlisting}

what changes while running the current instruction? \textbf{Try to guess and discuss!}
\end{frame}

\begin{frame}[fragile]
\texttt{\center
\begin{tabular}{| l | l | l | l |}
\hline
PC & S.Pose & S.Orientation & S.Position \\
\hline
4 & Sitting & Left & (0,3) \\
\hline
\end{tabular}
} \ \\ \ \\ \ \\

\begin{lstlisting}[frame=shadowbox,numbers=left]
take 3 steps forward 
sit on the chair 
turn left 
slide 3 steps forward
\end{lstlisting}

what changes while running the current instruction? \textbf{Try to guess and discuss!}
\end{frame}

\begin{frame}[fragile]
\texttt{\center
\begin{tabular}{| l | l | l | l |}
\hline
PC & S.Pose & S.Orientation & S.Position \\
\hline
END & Sitting & Left & (-3,3) \\
\hline
\end{tabular}
} \ \\ \ \\ \ \\

\begin{lstlisting}[frame=shadowbox,numbers=left]
take 3 steps forward 
sit on the chair 
turn left 
slide 3 steps forward
\end{lstlisting}

what do we do now? \textbf{Try to guess and discuss!}
\end{frame}

\SlideSubSection{A slight formalization}
\begin{slide}{
\item We say that an instruction \texttt{I} is a \textit{function} that, given a pair of \texttt{PC} and \texttt{S}, returns a new pair of \texttt{PC} and \texttt{S}
\pause
\item Do not panic now, math..y symbols incoming!
\pause
\item $(PC,S) \overset{Instr}{\rightarrow} (PC',S')$
}\end{slide}

\begin{slide}{
\item Consider instruction \texttt{sit on the chair} (we will shorten it to \texttt{sit})
\item \textit{How do we change the current instruction?}
\item \textit{How do we change the position of the resulting state depending on the orientation of the input state?}
}\end{slide}

\begin{slide}{
\item Consider instruction \texttt{sit on the chair} (we will shorten it to \texttt{sit})
\begin{itemize}
\small
\item $(PC,S) \overset{sit}{\rightarrow} (PC+1,S[Pose \mapsto Sitting])$
\end{itemize}
\item We increment the current instruction index by one
\item We change the pose of the resulting state independent on the input state
\begin{itemize}
\item $S[Pose \mapsto Sitting]$ is read as \textit{``S, where pose is sitting''}
\end{itemize}
}\end{slide}

\begin{slide}{
\item Consider instruction \texttt{stand up} (we will shorten it to \texttt{stand})
\item \textit{How do we change the current instruction?}
\item \textit{How do we change the position of the resulting state depending on the orientation of the input state?}
}\end{slide}

\begin{slide}{
\item Consider instruction \texttt{stand up} (we will shorten it to \texttt{stand})
\begin{itemize}
\small
\pause
\item $(PC,S) \overset{stand}{\rightarrow} (PC+1,S[Pose \mapsto Standing])$
\end{itemize}
\item We increment the current instruction index by one
\item We change the pose of the resulting state independent on the input state
}\end{slide}

\begin{slide}{
\item Consider instruction \texttt{take 3 steps forward} (we will shorten it to \texttt{fwd 3})
\item \textit{How do we determine the next instruction index?}
\item \textit{How do we change the position of the resulting state?}
\begin{itemize}
\item Are there dependencies from the input state?
\end{itemize}
}\end{slide}

\begin{frame}[fragile]
\texttt{\center
\begin{tabular}{| l | l | l | l |}
\hline
PC & S.Pose & S.Orientation & S.Position \\
\hline
104 & Standing & Left & (10,20) \\
\hline
\end{tabular}
} \ \\ \ \\ \ \\

\begin{lstlisting}[frame=shadowbox,numbers=left,firstnumber=103]
...
take 3 steps forward 
...
\end{lstlisting}

\texttt{\center
\begin{tabular}{| l | l | l | l |}
\hline
PC & S.Pose & S.Orientation & S.Position \\
\hline
\textcolor{UNL@Scarlet}{105} & Standing & Left & (\textcolor{UNL@Scarlet}{7},20) \\
\hline
\end{tabular}
}
\end{frame}

\begin{frame}[fragile]
\texttt{\center
\begin{tabular}{| l | l | l | l |}
\hline
PC & S.Pose & S.Orientation & S.Position \\
\hline
104 & Standing & Right & (10,20) \\
\hline
\end{tabular}
} \ \\ \ \\ \ \\

\begin{lstlisting}[frame=shadowbox,numbers=left,firstnumber=103]
...
take 3 steps forward 
...
\end{lstlisting}

\texttt{\center
\begin{tabular}{| l | l | l | l |}
\hline
PC & S.Pose & S.Orientation & S.Position \\
\hline
\textcolor{UNL@Scarlet}{105} & Standing & Right & (\textcolor{UNL@Scarlet}{13},20) \\
\hline
\end{tabular}
}
\end{frame}

\begin{slide}{
\item Consider instruction \texttt{take 3 steps forward} (we will shorten it to \texttt{fwd 3})
\begin{align*}
\tiny
\left\lbrace
\begin{array}{l l}
&(PC,S) \overset{fwd 3}{\rightarrow} (PC+1,S[Position \mapsto S.Position+(0,3)])\\
&when \; S.Orientation=Forward \\
&(PC,S) \overset{fwd 3}{\rightarrow} (PC+1,S[Position \mapsto S.Position-(0,3)]) \\
&when \; S.Orientation=Backward \\
&(PC,S) \overset{fwd 3}{\rightarrow} (PC+1,S[Position \mapsto S.Position+(3,0)]) \\
&when S.Orientation=Right \\
&(PC,S) \overset{fwd 3}{\rightarrow} (PC+1,S[Position \mapsto S.Position-(3,0)]) \\
&when S.Orientation=Left \\
\end{array}
\right.
\end{align*}
\item We always increment the instruction by one
\item We change the position of the resulting state depending on the orientation of the input state
}\end{slide}

\begin{slide}{
\item Consider instruction \texttt{if A then B else C}
\item \textit{How do we determine the next instruction index?}
\item \textit{How do we change the state?}
}\end{slide}

\begin{frame}[fragile]
\texttt{\center
\begin{tabular}{| l | l | l | l |}
\hline
PC & S.Pose & S.Orientation & S.Position \\
\hline
24 & Standing & Right & (10,20) \\
\hline
\end{tabular}
} \ \\ \ \\ \ \\

\begin{lstlisting}[frame=shadowbox,numbers=left,firstnumber=23]
...
if A is ``black'' then 
  turn left by 90 * B degrees 
otherwise 
  turn left by 90 * C degrees 
...
\end{lstlisting}

\texttt{\center
\begin{tabular}{| l | l | l | l |}
\hline
PC & S.Pose & S.Orientation & S.Position \\
\hline
\textcolor{UNL@Scarlet}{25\footnote{Assuming student's shirt is black}} & Standing & Right & (10,20) \\
\hline
\end{tabular}
}
\end{frame}

\begin{frame}[fragile]
\texttt{\center
\begin{tabular}{| l | l | l | l |}
\hline
PC & S.Pose & S.Orientation & S.Position \\
\hline
24 & Standing & Right & (10,20) \\
\hline
\end{tabular}
} \ \\ \ \\ \ \\

\begin{lstlisting}[frame=shadowbox,numbers=left,firstnumber=23]
...
if A is ``black'' then 
  turn left by 90 * B degrees 
otherwise 
  turn left by 90 * C degrees 
...
\end{lstlisting}

\texttt{\center
\begin{tabular}{| l | l | l | l |}
\hline
PC & S.Pose & S.Orientation & S.Position \\
\hline
\textcolor{UNL@Scarlet}{27\footnote{Assuming student's shirt is not black}} & Standing & Right & (10,20) \\
\hline
\end{tabular}
}
\end{frame}

\begin{slide}{
\item Consider instruction \texttt{if A then B else C} (shortened by as $if_{ABC}$)
\pause
\item We jump to the first instruction of the \texttt{B} block if the condition evaluates to \texttt{TRUE}
\pause
\item We jump to the first instruction of the \texttt{C} block if the condition evaluates to \texttt{FALSE}
\pause
\item We leave the state unchanged
\pause

$$
\tiny
\left\{
\begin{matrix}
(PC,S) \overset{if_{ABC}}{\rightarrow} (loc(B),S) &when& (PC,S) \overset{A}{\rightarrow} \mathtt{TRUE} \\
(PC,S) \overset{if_{ABC}}{\rightarrow} (loc(C),S) &when& (PC,S) \overset{A}{\rightarrow} \mathtt{FALSE} \\
\end{matrix}
\right.
$$

}\end{slide}

\begin{slide}{
\item Consider instruction \texttt{while A do B}
\item \textit{How do we determine the next instruction index?}
\item \textit{How do we change the state?}
}\end{slide}

\begin{frame}[fragile]
\texttt{\center
\begin{tabular}{| l | l | l | l |}
\hline
PC & S.Pose & S.Orientation & S.Position \\
\hline
24 & Standing & Right & (10,20) \\
\hline
\end{tabular}
} \ \\ \ \\ \ \\

\begin{lstlisting}[frame=shadowbox,numbers=left,firstnumber=23]
...
while A is ``sunny'' do
  order another beer
  enjoy the day for another hour
go back to work
...
\end{lstlisting}

\texttt{\center
\begin{tabular}{| l | l | l | l |}
\hline
PC & S.Pose & S.Orientation & S.Position \\
\hline
\textcolor{UNL@Scarlet}{25\footnote{As long as it is sunny}} & Standing & Right & (10,20) \\
\hline
\end{tabular}
}
\end{frame}

\begin{frame}[fragile]
\texttt{\center
\begin{tabular}{| l | l | l | l |}
\hline
PC & S.Pose & S.Orientation & S.Position \\
\hline
24 & Standing & Right & (10,20) \\
\hline
\end{tabular}
} \ \\ \ \\ \ \\

\begin{lstlisting}[frame=shadowbox,numbers=left,firstnumber=23]
...
while A is ``sunny'' do
  order another beer
  enjoy the day for another hour
go back to work
...
\end{lstlisting}

\texttt{\center
\begin{tabular}{| l | l | l | l |}
\hline
PC & S.Pose & S.Orientation & S.Position \\
\hline
\textcolor{UNL@Scarlet}{27\footnote{When it stops being sunny}} & Standing & Right & (10,20) \\
\hline
\end{tabular}
}
\end{frame}

\begin{slide}{
\item Consider instruction \texttt{while A do B} (shortened by as $while_{AB}$)
\pause
\item We jump to the first instruction of the \texttt{B} block if the condition evaluates to \texttt{TRUE}
\pause
\item We jump to after the last instruction of the \texttt{B} block if the condition evaluates to \texttt{FALSE}
\pause
\item We leave the state unchanged
\pause

$$
\tiny
\left\lbrace
\begin{array}{l l}
&(PC,S) \overset{while_{AB}}{\rightarrow} (loc(B),S) \; when \; (PC,S) \overset{A}{\rightarrow} \mathtt{TRUE} \\
&(PC,S) \overset{while_{AB}}{\rightarrow} (lastloc(B)+1,S) \; when \; (PC,S) \overset{A}{\rightarrow} \mathtt{FALSE} \\
\end{array}
\right.
$$

}\end{slide}

\begin{frame}{This is it!}
\center
\fontsize{18pt}{7.2}\selectfont
The best of luck, and thanks for the attention!
\end{frame}

\end{document}

\begin{slide}{
\item ...
}\end{slide}

\begin{frame}[fragile]
\begin{lstlisting}
...
\end{lstlisting}
\end{frame}


