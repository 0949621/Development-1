\documentclass{beamer}
\usetheme[hideothersubsections]{HRTheme}
\usepackage{beamerthemeHRTheme}
\usepackage[utf8]{inputenc}
\usepackage{graphicx}
\usepackage[space]{grffile}
\usepackage{listings}
\lstset{language=C,
basicstyle=\ttfamily\footnotesize,
mathescape=true,
breaklines=true}
\lstset{
  literate={ï}{{\"i}}1
           {ì}{{\`i}}1
}
\usepackage{color}
\newcommand{\red}[1]{
\textcolor{red}{#1}
}
\newcommand{\ts}{\textbackslash}

\title{Looping and iteration}

\author{The INFDEV Team @ HR}

\institute{Hogeschool Rotterdam \\ 
Rotterdam, Netherlands}

\date{}

\begin{document}
\maketitle

\SlideSection{Introduction}
\SlideSubSection{Lecture topics}
\begin{slide}{
\item Repeated behaviors
\item \texttt{while} statements and their semantics
\item Expressive power of \texttt{while}
\item Termination and infinite iteration
\item Explosion of states with \texttt{while}
}\end{slide}


\SlideSection{Repeated behaviors}
\SlideSubSection{Repeated behaviors}
\begin{slide}{
\item Sometimes running code just once is not enough
\item We can \textit{loop} execution of a block of code until some \textit{condition} is met
\item Extreme increase in expressive power
}\end{slide}

\begin{slide}{
\item While \textit{there are hostile aliens}
\item Do \textit{attack an alien}
}\end{slide}

\begin{slide}{
\item Loops can solve very big problems
\item Each step of the loop removes a part of the problem
\item We typically stop when all parts of the problem have been removed
}\end{slide}

\SlideSubSection{Breaking problems up}
\begin{slide}{
\item Problem: \textbf{kill all aliens}
\item Problem piece: \textbf{a single alien to be killed}
\item Solution piece: \textbf{attack a single alien}
\item Termination condition: \textbf{there are no more aliens}
}\end{slide}

\begin{slide}{
\item Of course loops can be combined with each other
\item This means that we can \textit{cascade} repetition
\item This is the building block of \textit{intelligent decisions} in our programs
}\end{slide}

\begin{slide}{
\item While \textit{there are hostile alien armies}
\begin{itemize}
\item Do \textit{pick an alien army}
\item While \textit{there are aliens in the army}
\begin{itemize}
\item Do \textit{attack an alien}
\end{itemize}
\end{itemize}
}\end{slide}

\SlideSection{Repeating behaviors in Python}
\SlideSubSection{\texttt{while}}
\begin{slide}{
\item Python offers built-in facilities for repetition
\item \texttt{while} statement
\item We can repeat execution of a block of code
}\end{slide}

\begin{slide}{
\item The general form is \texttt{while CONDITION: BODY} ($w_{CB}$)
\item If the condition is true, then we jump to the beginning of \texttt{BODY}, otherwise we jump to the end of the whole \texttt{while}
$$
\tiny
\left\{
\begin{matrix}
(PC,S) \overset{w_{CB}}{\rightarrow} (firstLine(B),S) &when& (PC,S) \overset{C}{\rightarrow} TRUE & \\
(PC,S) \overset{w_{CB}}{\rightarrow} (skipAfter(B),S) &when& (PC,S) \overset{C}{\rightarrow} FALSE & \\
\end{matrix}
\right.
$$
}\end{slide}

\begin{slide}{
\item Remember that Python is \textit{indentation}-based
\item White-spaces go at the beginning of some lines
\item A more indented line is \textit{within} a less indented line above
}\end{slide}

\begin{slide}{
\item Indentation specifies where the \texttt{body} begins and ends
\item The general form of a \texttt{while} is thus:
\begin{itemize}
\item \texttt{while COND:}
\item \texttt{newline}
\item \textbf{indentation}
\item \texttt{code of body}
\item \textbf{de-indentation}
\end{itemize}
}\end{slide}

\begin{frame}[fragile]{A correct example}
\begin{lstlisting}[frame=shadowbox]
n = 64
i = 0
while n > 1:
  n = n / 2
  i = i + 1
\end{lstlisting}
\end{frame}

\begin{frame}[fragile]{An incorrect example}
\begin{lstlisting}[frame=shadowbox]
n = 64
i = 0
while n > 1:
n = n / 2
i = i + 1
\end{lstlisting}
\end{frame}

\begin{slide}{
\item \texttt{while} statements eventually terminate (hopefully)
\item if the \texttt{condition} evaluates to \texttt{False}, then we skip after the end of the \texttt{while}
}\end{slide}

\begin{frame}[fragile]{After a \texttt{while}}
\begin{lstlisting}[frame=shadowbox]
n = 64
i = 0
while n > 1:
  n = n / 2
  i = i + 1
print(i)
\end{lstlisting}
\end{frame}

\begin{frame}[fragile]{After a \texttt{while}?}
Without indentation, this:
\begin{lstlisting}[frame=shadowbox,basicstyle=\ttfamily\tiny]
n = 64
i = 0
while n > 1:
n = n / 2
i = i + 1
print(i)
\end{lstlisting}

would be indistinguishable from both:

\begin{tabular}{| c | c |}
\hline
\begin{lstlisting}[basicstyle=\ttfamily\tiny]
n = 64
i = 0
while n > 1:
  n = n / 2
  i = i + 1
print(i)
\end{lstlisting}
&
\begin{lstlisting}[basicstyle=\ttfamily\tiny]
n = 64
i = 0
while n > 1:
  n = n / 2
  i = i + 1
  print(i)
\end{lstlisting} \\
\hline
\end{tabular}
\end{frame}

\SlideSection{Reasoning about \texttt{while}}
\begin{slide}{
\item \texttt{while} effectively rewrites the code to become as long as the problem needs
\item Until run-time, we are not really sure how long the code will become
}\end{slide}

\begin{frame}[fragile]{Example \texttt{while}'s}
What values of \texttt{n} will we \texttt{print}?
\begin{lstlisting}[frame=shadowbox,basicstyle=\ttfamily\tiny]
i = 0
j = 1
n = 0
while i < 10:
  i = i + 1
  n = n + j
  print(n),
\end{lstlisting}

\pause

\ \\

\textbf{1 2 3 4 5 6 7 8 9 10}
\end{frame}

\begin{frame}[fragile]{Example \texttt{while}'s}
What values of \texttt{n} will we \texttt{print}?
\begin{lstlisting}[frame=shadowbox,basicstyle=\ttfamily\tiny]
i = 0
j = 2
n = 0
while i < 10:
  i = i + 1
  n = n + j
  print(n),
\end{lstlisting}

\pause

\ \\

\textbf{2 4 6 8 10 12 14 16 18 20}
\end{frame}

\begin{slide}{
\item Each iteration produces new values of the variables
\item These new values are then fed back into the next iteration
\item Eventually these cause the condition to become false
\begin{itemize}
\item Or the program runs forever and never produces a result
\pause
\item We'd rather not have this one
\end{itemize}
}\end{slide}

\begin{frame}[fragile]{Example \texttt{while}'s}
\texttt{\center
\begin{tabular}{| l | l | l |}
\hline
i & j & n \\
\hline
0 & 2 & 0 \\
\hline
\end{tabular}
} \ \\ \ \\ \ \\

\begin{lstlisting}[frame=shadowbox,basicstyle=\ttfamily\tiny]
while i < 10:
  i = i + 1
  n = n + j
  print(n),
\end{lstlisting}

\pause

\begin{tabular}{| l | l | l |}
\hline
i & j & n \\
\hline
\red{1} & 2 & \red{2} \\
\hline
\end{tabular}
\end{frame}

\begin{frame}[fragile]{Example \texttt{while}'s}
\texttt{\center
\begin{tabular}{| l | l | l |}
\hline
i & j & n \\
\hline
1 & 2 & 2 \\
\hline
\end{tabular}
} \ \\ \ \\ \ \\

\begin{lstlisting}[frame=shadowbox,basicstyle=\ttfamily\tiny]
while i < 10:
  i = i + 1
  n = n + j
  print(n),
\end{lstlisting}

\pause

\begin{tabular}{| l | l | l |}
\hline
i & j & n \\
\hline
\red{2} & 2 & \red{4} \\
\hline
\end{tabular}
\end{frame}

\begin{frame}[fragile]{Example \texttt{while}'s}
\texttt{\center
\begin{tabular}{| l | l | l |}
\hline
i & j & n \\
\hline
2 & 2 & 4 \\
\hline
\end{tabular}
} \ \\ \ \\ \ \\

\begin{lstlisting}[frame=shadowbox,basicstyle=\ttfamily\tiny]
while i < 10:
  i = i + 1
  n = n + j
  print(n),
\end{lstlisting}

\pause

\begin{tabular}{| l | l | l |}
\hline
i & j & n \\
\hline
\red{3} & 2 & \red{6} \\
\hline
\end{tabular}
\end{frame}

\begin{frame}[fragile]{Example \texttt{while}'s}
After \texttt{k} iterations (for \texttt{k < 10}): \\


\texttt{\center
\begin{tabular}{| l | l | l |}
\hline
i & j & n \\
\hline
k & 2 & 2 * k \\
\hline
\end{tabular}
} \ \\ \ \\ \ \\

\begin{lstlisting}[frame=shadowbox,basicstyle=\ttfamily\tiny]
while i < 10:
  i = i + 1
  n = n + j
  print(n),
\end{lstlisting}

\pause

\begin{tabular}{| l | l | l |}
\hline
i & j & n \\
\hline
\red{k+1} & 2 & \red{2 * k + 2} \\
\hline
\end{tabular}
\end{frame}

\begin{frame}[fragile]{Example \texttt{while}'s}
After \texttt{k} iterations (for \texttt{k = 10}): \\


\texttt{\center
\begin{tabular}{| l | l | l |}
\hline
i & j & n \\
\hline
10 & 2 & 20 \\
\hline
\end{tabular}
} \ \\ \ \\ \ \\

\begin{lstlisting}[frame=shadowbox,basicstyle=\ttfamily\tiny]
while i < 10:
  i = i + 1
  n = n + j
  print(n),
\end{lstlisting}

\pause

We jump to the first instruction \textbf{after} the while loop, and do not touch the state further.

\begin{tabular}{| l | l | l |}
\hline
i & j & n \\
\hline
10 & 2 & 20 \\
\hline
\end{tabular}
\end{frame}

\SlideSubSection{Readability and termination}
\begin{slide}{
\item The loop above is \textit{well designed}
\item All iterations produce a new piece of a logical series
\begin{itemize}
\item (The \texttt{j}-th row of the table of multiplication)
\end{itemize}
}\end{slide}

\begin{slide}{
\item After each iteration we know that we have \texttt{i} elements correctly computed
\item After each iteration we know that we have \texttt{10-i} elements still to compute
\item When \texttt{i=10} then we have \texttt{10-10} elements still to compute
\begin{itemize}
\item This is the \textbf{termination condition}
\item Since \texttt{i} keeps growing, we know that eventually the termination condition will be met
\end{itemize}
}\end{slide}

\SlideSubSection{Nesting}
\begin{slide}{
\item Loops can be nested
\item A loop can be inside a loop (which can further be inside other loops)
\item This makes it slightly harder to reason
}\end{slide}

\begin{frame}[fragile]{Example nested \texttt{while}'s}
\begin{lstlisting}[frame=shadowbox,basicstyle=\ttfamily\tiny]
j = 1
while j <= 10:
  i = 0
  n = 0
  while i < 10:
    i = i + 1
    n = n + j
    print(n),
    print("\t"),
  j = j + 1
  print("")
\end{lstlisting}
\end{frame}

\begin{frame}[fragile]{Example nested \texttt{while}'s}
We now know that the semantics of the inner loop is \textbf{print the j-th row of the table of multiplication}, so instead of reasoning on: \\ \ \\

\begin{lstlisting}[frame=shadowbox,basicstyle=\ttfamily\tiny]
j = 1
while j <= 10:
  i = 0
  n = 0
  while i < 10:
    i = i + 1
    n = n + j
    print(n),
    print("\t"),
  j = j + 1
  print("")
\end{lstlisting}

we reason on: \\ \ \\

\begin{lstlisting}[frame=shadowbox,basicstyle=\ttfamily\tiny]
j = 1
while j <= 10:
  print the j-th row of the table of multiplication
  j = j + 1
  print("")
\end{lstlisting}

\end{frame}

\begin{frame}[fragile]{Example nested \texttt{while}'s}
\texttt{\center
\begin{tabular}{| l | l |}
\hline
j & output \\
\hline
1 & nothing \\
\hline
\end{tabular}}
 \ \\ \ \\ \ \\

\begin{lstlisting}[frame=shadowbox,basicstyle=\ttfamily\tiny]
while j <= 10:
  print the j-th row of the table of multiplication
  j = j + 1
  print("")
\end{lstlisting}

\pause 

\texttt{\center
\begin{tabular}{| l | l |}
\hline
j & output \\
\hline
\red{2} & \red{1st row of table} \\
\hline
\end{tabular}}
\end{frame}

\begin{frame}[fragile]{Example nested \texttt{while}'s}
\texttt{\center
\begin{tabular}{| l | l |}
\hline
j & output \\
\hline
2 & 1st row of table \\
\hline
\end{tabular}}
 \ \\ \ \\ \ \\

\begin{lstlisting}[frame=shadowbox,basicstyle=\ttfamily\tiny]
while j <= 10:
  print the j-th row of the table of multiplication
  j = j + 1
  print("")
\end{lstlisting}

\pause 

\texttt{\center
\begin{tabular}{| l | l |}
\hline
j & output \\
\hline
\red{3} & \red{1st and 2nd rows of table} \\
\hline
\end{tabular}}
\end{frame}

\begin{frame}[fragile]{Example nested \texttt{while}'s}
...
\end{frame}

\begin{frame}[fragile]{Example nested \texttt{while}'s}
Thus for all \texttt{k}'s such that \texttt{k} $\leq 10$:

\texttt{\center
\begin{tabular}{| l | l |}
\hline
j & output \\
\hline
k & first k-1 rows of table \\
\hline
\end{tabular}}
 \ \\ \ \\ \ \\

\begin{lstlisting}[frame=shadowbox,basicstyle=\ttfamily\tiny]
while j <= 10:
  print the j-th row of the table of multiplication
  j = j + 1
  print("")
\end{lstlisting}

\pause 

\texttt{\center
\begin{tabular}{| l | l |}
\hline
j & output \\
\hline
\red{k+1} & \red{first k rows of table} \\
\hline
\end{tabular}}
\end{frame}

\begin{frame}[fragile]{Example nested \texttt{while}'s}
Eventually we get \texttt{j = 11}:

\texttt{\center
\begin{tabular}{| l | l | l |}
\hline
j & output & PC \\
\hline
11 & first 10 rows of table & 1 \\
\hline
\end{tabular}}
 \ \\ \ \\ \ \\

\begin{lstlisting}[frame=shadowbox,basicstyle=\ttfamily\tiny,numbers=right]
while j <= 10:
  print the j-th row of the table of multiplication
  j = j + 1
  print("")
...
\end{lstlisting}

\pause 

\texttt{\center
\begin{tabular}{| l | l | l |}
\hline
j & output & PC \\
\hline
11 & first 10 rows of table & \red{5} \\
\hline
\end{tabular}}
\end{frame}

\SlideSection{Termination (or lack thereof)}
\SlideSubSection{Wait! It gets worse!}
\begin{slide}{
\item It is not guaranteed that a loop will terminate
\item A loop that does not terminate gets the program stuck forever
\item It is a bit sad for the machine
\item Care when designing loops is needed to prevent this
}\end{slide}

\SlideSubSection{Care in the design}
\begin{slide}{
\item A loop changes the state of the program many times
\item A \textbf{good} loop changes the state in one \textbf{direction}
\item Every step should bring us closer to the \textbf{final state}
\item The condition defines the aspects of the final state
}\end{slide}

\begin{frame}[fragile]{Example non terminating \texttt{while}}
\begin{lstlisting}[frame=shadowbox,basicstyle=\ttfamily\tiny]
i = 1
while i > 0:
  i = i + 1
  print(i)
\end{lstlisting}

\textbf{What is the issue here?}

\pause

Iterations do not go \textbf{towards} the condition, but \textbf{away} from it.

\end{frame}

\begin{frame}[fragile]{Example non terminating \texttt{while}}
This is not a duplicated slide.

\begin{lstlisting}[frame=shadowbox,basicstyle=\ttfamily\tiny]
ì = 1
while i < 10:
  ì = ì + 1
  print(ì)
\end{lstlisting}

\textbf{What is the issue here?}

\pause

Iterations are \textbf{orthogonal (unrelated)} to the condition.

No iteration changes elements tested in the condition.
\end{frame}

\SlideSection{Exponential explosion of potential control-paths}
\SlideSubSection{Exponential explosion of potential control-paths}
\begin{slide}{
\item How many things can happen in a \texttt{while} loop?
\item Depends on its content
}\end{slide}

\begin{slide}{
\item The more we nest loops and conditionals within a loop...
\pause
\item ...the more things may happen at run-time
}\end{slide}

\begin{frame}[fragile]{Loops with nested \texttt{if}'s}
\begin{lstlisting}[frame=shadowbox,basicstyle=\ttfamily\tiny]
while C0:
  if C1:
    A1
  else:
    B1
\end{lstlisting}

How many execution paths per \textbf{N} iterations of the loop? \\ \ \\

\pause

\textbf{2} execution paths per iteration \\
\textbf{$2^N$} execution paths per \texttt{N} iterations \\
\end{frame}

\begin{frame}[fragile]{Loops with nested \texttt{if}'s}
\begin{lstlisting}[frame=shadowbox,basicstyle=\ttfamily\tiny]
while C0:
  if C1:
    A1
  else:
    B1
\end{lstlisting}

\textbf{Example:} execution paths per \textbf{3} iterations of the loop. \\

\texttt{A1 A1 A1} \\
\texttt{A1 A1 B1} \\
\texttt{A1 B1 A1} \\
\texttt{A1 B1 B1} \\
\texttt{B1 A1 A1} \\
\texttt{B1 A1 B1} \\
\texttt{B1 B1 A1} \\
\texttt{B1 B1 B1} \\

\end{frame}

\begin{slide}{
\item Consider a loop that performs \texttt{m} iterations
\item With \texttt{n} \texttt{if}'s inside
\item Each iteration can have $2^n$ possible execution paths
\item The whole loop can have $2^{n\times m}$ possible execution paths\footnote{$2^{2\times 10} = 1048576$}
}\end{slide}

\begin{slide}{
\item Each path can alter the state in a different way
\item After a \texttt{while} with many billions possible paths
\begin{itemize}
\item We have many billions possible resulting states
\end{itemize}
}\end{slide}

\begin{slide}{
\item The more nested the code inside a \texttt{while}
\item The more complex its behavior
\item \textit{The harder it is to reason about it!}
}\end{slide}

\SlideSubSection{The value of reasoning}
\begin{slide}{
\item \textbf{Always keep in mind:}
\item You have the power to make your own life a living Hell...
\pause
\item ...unless you reason first and then structure code logically
}\end{slide}

\SlideSection{Assignment}
\SlideSubSection{Drawing with loops!}
\begin{slide}{
\item Exclusively with \texttt{while}-loops
\item Reading the appropriate parameters from the command line (figure height, width, etc.)
\item Draw the following figures on the console:
\begin{itemize}
\item A full square 
\item The border of a square
\item A rectangle triangle
\item A centered, isosceles triangle
\item A circle
\item A smiley face
\end{itemize}
\item \textbf{See assignment description online for examples and better details}
}\end{slide}

\begin{frame}{This is it!}
\center
\fontsize{18pt}{7.2}\selectfont
The best of luck, and thanks for the attention!
\end{frame}

\end{document}

\begin{slide}{
\item ...
}\end{slide}

\begin{frame}[fragile]
\begin{lstlisting}
...
\end{lstlisting}
\end{frame}
