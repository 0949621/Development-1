\documentclass{beamer}
\usetheme[hideothersubsections]{HRTheme}
\usepackage{beamerthemeHRTheme}
\usepackage[utf8]{inputenc}
\usepackage{graphicx}
\usepackage[space]{grffile}
\usepackage{comment}
\title{Functions}


\author{TEAM INFDEV}

\institute{Hogeschool Rotterdam \\ 
Rotterdam, Netherlands}

\date{}

\begin{document}
\maketitle

\SlideSection{Introduction}
\SlideSubSection{Lecture topics}
\begin{slide}{
\item Mechanism of abstraction
\item The need for functions
\item Functions in Python
}\end{slide}

\SlideSection{What is abstraction?}
\SlideSubSection{Introduction}
\begin{slide}{
\item The big issue of the whole course is \textbf{abstraction} in programming
\item Abstraction is a fundamental concept in programming to reduce repetition
\item We sit atop a mountain of abstraction, which we make taller at every iteration
}\end{slide}

\SlideSubSection{Grab the student next to you}
\begin{slide}{
\item Describe what you just did so that someone else can perform the same action
\pause
\item Now add specific details about the movements of your arm and phalanges (pieces of fingers)
\pause
\item Now realize that there are even more subcomponents: individual muscles, tendons, etc.
\pause
\item But then we have also cells that make these up
\item ...
}\end{slide}

\SlideSubSection{Human love for abstraction}
\begin{slide}{
\item Our brain cannot handle so many details
\item To cope with this, we are structured in layers
\item Our consciousness manipulates only the upper layers with simple instructions
\item \textit{Raise arm above head}
}\end{slide}

\begin{slide}{
\item The same happens with regular language
\item ``\textit{Go buy a liter of milk}'' is quite a short description
\item The underlying operation is very complex
}\end{slide}

\begin{frame}[fragile]{Complexity of simple instructions}
\begin{lstlisting}
Go buy a liter of milk =
  Turn game off
  Get up from the couch
  Curse the instruction giver
  Get dressed
  Put money in pocket
  Leave house
  Reach nearest shop
  Enter shop
  Find milk
  Take one liter bottle
  Pay milk
  Go home
  Give milk to instruction giver
\end{lstlisting}
\end{frame}

\begin{slide}{
\item And clearly something like ``\textit{reach nearest shop}'' is not a trivial instruction by itself
\item Think about all the things you give for granted
\begin{itemize}
\item Crossing roads
\item Traffic lights
\item Pathfinding
\item Road work and obstructions
\item Use of transportation methods
\item ...
\end{itemize}
}\end{slide}

\SlideSection{Problem discussion}
\SlideSubSection{Introduction}
\begin{slide}{
\item Consider many operations on number
\begin{itemize}
\item finding whether a number is prime
\item finding whether a number is odd or even
\item finding the Fibonacci value of a given number
\item ...
\end{itemize}
}\end{slide}

\begin{frame}[fragile]{counting down..}
\begin{codewithblock}{\pause \item What does \texttt{cnt} contain if \texttt{startAt} equals 10? \item What do we do with the value of startAt? \item Does it even matter?}
cnt = startAt
while not(cnt == 0):
  cnt = cnt - 1
print("Lift off!!!")
\end{codewithblock}
\end{frame}

\begin{slide}{
\item Suppose that we want another start, \texttt{newStartAt}
\item How do we do it?
}\end{slide}

\begin{frame}[fragile]{Lenght of a list}
\begin{codewithblock}{\pause \item Looks suspiciously like the previous code block \item Why?}

cnt = newStartAt
while not(cnt == 0):
  cnt = cnt - 1
print("Lift off!!!")
\end{codewithblock}
\end{frame}

\SlideSection{General idea}
\SlideSubSection{Adding our own layers}
\begin{slide}{
\item The goal of this lecture is to add a new layer of abstraction to our programs
\item We wish to reuse \textbf{implementations}
\item This layer of abstraction is called \textbf{functions}
}\end{slide}

\begin{frame}[fragile]{Adding our own layers}
\begin{lstlisting}
+-------------+
| ...         |
+------------------+
| Functions        |
+-----------------------+
| data structures       |
+----------------------------+
| if, for, while, variables  |
+--------------------------------+
| (Python) runtime               |
+--------------------------------+
...
\end{lstlisting}
\end{frame}

\SlideSubSection{Description}
\begin{slide}{
\item A function is a collection of instructions and variables
\item Some instructions and variables are fixed inside its \textbf{body}
\item Other instructions and variables come from outside the function, and thus are not fixed; these are called \textbf{parameters} of the function
\item We try to strike the right balance between flexibility and work done
\item The function returns a final result that can be recovered by the code that uses the function
}\end{slide}

\begin{frame}[fragile]{Blueprint of a function (NOT ACTUAL PYTHON CODE!)}
\begin{codewithblock}{\pause \item \texttt{count\_down} is the \textbf{function name} \item \texttt{number} is the only \textbf{parameter} \item Lines 2 through 4 are \textbf{fixed} \item \texttt{"Lift off!!!"} is the \textbf{final result}}
count_down starting from a number:
  cnt = number
  while not(cnt == 0):
   cnt = cnt - 1
  return "Lift off!!!" as final result
\end{codewithblock}
\end{frame}

\SlideSubSection{Using the function}
\begin{slide}{
\item Code that needs a count down function can now simply invoke function \texttt{count\_down}
\item The resulting code will simply be \texttt{result = count\_down(5)}
\item \texttt{result} will be assigned with the value returned by the function, i.e., "Lift off!!!"
}\end{slide}

\SlideSection{Technical details}
\SlideSubSection{Introduction}
\begin{slide}{
\item A function can be defined in Python quite easily
\item The syntax is:
\begin{itemize}
\item \texttt{def <<name>>(<<parameters>>):}\footnote{Parameters might be none, thus we can write simply \texttt{()}}\footnote{Multiple parameters are separated by a comma, thus \texttt{(<<p1>>,<<p2>>,...,<<pn>>)}}
\begin{itemize}
\item \texttt{body}
\item \texttt{return <<result>>}
\end{itemize}
\end{itemize}
\item Inside a function we can put whatever instructions we need
\begin{itemize}
\item \texttt{if}
\item \texttt{for}
\item ...
\end{itemize}
}\end{slide}

\SlideSubSection{Using the function}
\begin{slide}{
\item After we declare a function, we can use it
\item The syntax is quite simple
\begin{itemize}
\item \texttt{<<name>>(<<parameters>>)} to just call the function and ignore the result
\item \texttt{<<v>> = <<name>>(<<parameters>>)} to call the function and assign the result to the \texttt{<<v>>} variable
\end{itemize}
\item After calling the function, we enter the local environment of the function
\item Variables, the \texttt{PC}, etc. are separate from those of the calling site
}\end{slide}

\begin{frame}[fragile]{Runtime example}
\begin{statetable}
{|c|}
{PC}
{6}
\end{statetable} \ \\

\begin{lstlisting}
def count_down(number):
  cnt = number
  while not(cnt == 0):
    cnt = cnt - 1
  return "Lift off!!!"
print(count_down(2))
\end{lstlisting}

\pause

\begin{statetable}
	{|c|c|c|c|}
	{PC & count\_down & PC & number}
	{6 & nil & 2 & 2 }
\end{statetable} \ \\

\end{frame}

\begin{frame}[fragile]{Runtime example}
\begin{statetable}
	{|c|c|c|c|}
	{PC & count\_down & PC & number}
	{6 & nil & 2 & 2 }
\end{statetable} \ \\

\begin{lstlisting}
def count_down(number):
  cnt = number
  while not(cnt == 0):
    cnt = cnt - 1
  return "Lift off!!!"
print(count_down(2))
\end{lstlisting}


\pause

\begin{statetable}
	{|c|c|c|c|c|}
	{PC & count\_down & PC & number & cnt}
	{6 & nil & \red{3} & 2 & \red{2}}
\end{statetable} \ \\
\end{frame}

\begin{frame}[fragile]{Runtime example}
\begin{statetable}
	{|c|c|c|c|c|}
	{PC & count\_down & PC & number & cnt}
	{6 & nil & 3 & 2 & 2}
\end{statetable} \ \\

\begin{lstlisting}
def count_down(number):
  cnt = number
  while not(cnt == 0):
    cnt = cnt - 1
  return "Lift off!!!"
print(count_down(2))
\end{lstlisting}

\pause

\begin{statetable}
	{|c|c|c|c|c|}
	{PC & count\_down & PC & number & cnt}
	{6 & nil & \red{4} & 2 & 2}
\end{statetable} \ \\
\end{frame}


\begin{frame}[fragile]{Runtime example}
	\begin{statetable}
	{|c|c|c|c|c|}
	{PC & count\_down & PC & number & cnt}
	{6 & nil & 4 & 2 & 2}
	\end{statetable} \ \\
	
\begin{lstlisting}
def count_down(number):
  cnt = number
  while not(cnt == 0):
    cnt = cnt - 1
  return "Lift off!!!"
print(count_down(2))
\end{lstlisting}
	
	\pause
	
	\begin{statetable}
		{|c|c|c|c|c|}
		{PC & count\_down & PC & number & cnt}
		{6 & nil & \red{3} & 2 & \red{1}}
	\end{statetable} \ \\
\end{frame}

\begin{frame}[fragile]{Runtime example}
\textit{After a few steps...} \\

	\begin{statetable}
		{|c|c|c|c|c|}
		{PC & count\_down & PC & number & cnt}
		{6 & nil & \red{5} & 2 & \red{0}}
	\end{statetable} \ \\

\begin{lstlisting}
def count_down(number):
  cnt = number
  while not(cnt == 0):
    cnt = cnt - 1
  return "Lift off!!!"
print(count_down(2))
\end{lstlisting}

\pause
\textbf{Do we still need all the local variables of the function?}
\pause
\textbf{Where do we put the result?}
\pause

\begin{statetable}
	{|c|c|}
	{PC & count\_down }
	{6 & "Lift off!!!" }
\end{statetable} \ \\
\end{frame}

\SlideSubSection{Syntax and semantics}
\begin{slide}{
\item We will now describe how Python functions work precisely
\item This is a \textbf{fundamental} bit of knowledge that determines if you really do learn how to program or not
\item This \textbf{absolutely requires} a lot of focus to get
\item Please panic a bit on the inside
}\end{slide}

\SlideSubSection{Subtleties that make functions ``fun'' to use}
\begin{slide}{
\item About variables
\begin{itemize}
\item Variables and parameters inside a function have precise \textbf{scope} (visibility)
\item Primitive values given as parameters can be \textbf{changed only locally} to the function
\item References given as parameters can be \textbf{permanently changed} from within the function
\item Global variables defined outside the function may be \textbf{read but not changed} from within the function\footnote{\textbf{Unless you use some tricks we strongly discourage}}
\end{itemize}
\item About behaviour
\begin{itemize}
\item A function may \textbf{call itself}, in a process known as \textbf{recursion}
\item A function may \textbf{get as parameters and return other functions}, in a process known as \textbf{higher order functions}
\end{itemize}
}\end{slide}

\SlideSubSection{Local and global variables (basics of scope)}
\begin{slide}{
\item The parameters of a function are added to the list of accessible variables
\item They are only visible from inside the function
\item Global variables are also visible from inside the function
}\end{slide}

\begin{slide}{
\item Every call to a function generates a new value of the stack memory \texttt{S}
\item This contains (private copy of) all local variables
\item The original stack memory (the \textbf{global variables}) remains accessible, just read-only 
}\end{slide}

\begin{slide}{
\item Every call to a function also reserves some special locations in the stack
\item The local \texttt{PC} of the function
\item The local variables of the function
\item The returned value when the function is done
}\end{slide}

\begin{frame}[fragile]{Locals and globals}
\begin{codewithblock}{\item \texttt{x} is a global variable, visible outside and inside the function \item \texttt{z} is a local variable, visible only inside the function \pause \item \textbf{What does this program print?} \pause \item \texttt{10}, \texttt{30}, \texttt{20}}
x = 1

def f(z):
  return x * z

print(f(10))
print(f(30))
x = 2
print(f(10))
\end{codewithblock}
\end{frame}

\begin{frame}[fragile]{Locals and globals}
\begin{statetable}
{|c|}{PC}{1}
\end{statetable} \ \\

\begin{lstlisting}
x = 1

def f(z):
  return x * z

print(f(10))
x = 2
print(f(10))
\end{lstlisting}

\pause

\begin{statetable}
{|c|c|}{PC & x}{\red{6} & \red{1}}
\end{statetable} \ \\
\end{frame}

\begin{frame}[fragile]{Locals and globals}
\begin{statetable}
{|c|c|}{PC & x}{\red{6} & \red{1}}
\end{statetable} \ \\

\begin{lstlisting}
x = 1

def f(z):
  return x * z

print(f(10))
x = 2
print(f(10))
\end{lstlisting}

\pause

\begin{statetable}
{|c|c|c|c|c|}{PC & x & f & PC & z}{6 & 1 & \red{nil} & \red{4} & \red{10}}
\end{statetable} \ \\
\end{frame}

\begin{frame}[fragile]{Locals and globals}
\begin{statetable}
{|c|c|c|c|c|}{PC & x & f & PC & z}{6 & 1 & nil & 4 & 10}
\end{statetable} \ \\

\begin{lstlisting}
x = 1

def f(z):
  return x * z

print(f(10))
x = 2
print(f(10))
\end{lstlisting}

\pause

\begin{statetable}
{|c|c|c|}{PC & x & f}{\red{7} & 1 & 10}
\end{statetable} \ \\
\end{frame}

\begin{frame}[fragile]{Locals and globals}
\begin{statetable}
{|c|c|c|}{PC & x & f}{7 & 1 & 10}
\end{statetable} \ \\

\begin{lstlisting}
x = 1

def f(z):
  return x * z

print(f(10))
x = 2
print(f(10))
\end{lstlisting}

\pause

\begin{statetable}
{|c|c|}{PC & x}{\red{8} & \red{2}}
\end{statetable} \ \\
\end{frame}

\begin{frame}[fragile]{Locals and globals}
\begin{statetable}
{|c|c|c|c|c|}{PC & x & f & PC & z}{8 & 2 & \red{nil} & \red{4} & 10}
\end{statetable} \ \\

\begin{lstlisting}
x = 1

def f(z):
  return x * z

print(f(10))
x = 2
print(f(10))
\end{lstlisting}

\pause

\begin{statetable}
{|c|c|c|}{PC & x & f}{\red{8} & 2 & \red{20}}
\end{statetable} \ \\
\end{frame}

\begin{frame}[fragile]{Locals and globals}
\begin{codewithblock}{\item \texttt{x} is a global variable, visible outside and inside the function \item \texttt{z} is a local variable, visible only inside the function \pause \item \textbf{What does this program do?} \pause \item Crash with \texttt{NameError: name 'z' is not defined}}
x = 1

def f(z):
  return x * z

print(f(10))
x = 2
print(f(10))
print(z)
\end{codewithblock}
\end{frame}

\begin{frame}[fragile]{Locals and globals}
\begin{codewithblock}{\item \texttt{z} is a local variable, visible only inside the function \pause \item \textbf{What does this program print?} \pause \item \texttt{22}, \texttt{62}}
def f(z):
  z = z + 1
  return z * 2

print(f(10))
print(f(30))
\end{codewithblock}
\end{frame}

\SlideSubSection{Shadowing}
\begin{slide}{
\item The parameters of a function have priority over globals
\item They supersede global variables of the same name
}\end{slide}

\begin{frame}[fragile]{Shadowing}
\begin{codewithblock}{\item \texttt{x} is a global variable, potentially visible inside the function \item \texttt{x} is also a local variable of the function, which has priority over the global \texttt{x} \pause \item \textbf{What does this program print?} \pause \item \texttt{20}, \texttt{40}}
x = 1

def f(x):
  return x * 2

print(f(10))
print(f(20))
\end{codewithblock}
\end{frame}

\begin{frame}[fragile]{Shadowing}
\begin{statetable}
{|c|c|}{PC & x}{6 & 1}
\end{statetable} \ \\

\begin{lstlisting}
x = 1

def f(x):
  return x * 2

print(f(10))
print(f(20))
\end{lstlisting}

\pause

\begin{statetable}
{|c|c|c|c|c|}{PC & x & f & PC & x}{6 & 1 & \red{nil} & \red{4} & \red{10}}
\end{statetable} \ \\
\end{frame}

\begin{frame}[fragile]{Shadowing}
\begin{statetable}
{|c|c|c|c|c|}{PC & x & f & PC & x}{6 & 1 & nil & 4 & 10}
\end{statetable} \ \\

\begin{lstlisting}
x = 1

def f(x):
  return x * 2

print(f(10))
print(f(20))
\end{lstlisting}

\pause

\begin{statetable}
{|c|c|c|}{PC & x & f}{7 & 1 & \red{20}}
\end{statetable} \ \\
\end{frame}

\begin{frame}[fragile]{Shadowing}
\begin{statetable}
{|c|c|c|}{PC & x & f}{7 & 1 & 20}
\end{statetable} \ \\

\begin{lstlisting}
x = 1

def f(x):
  return x * 2

print(f(10))
print(f(20))
\end{lstlisting}

\pause

\begin{statetable}
{|c|c|c|c|c|}{PC & x & f & PC & x}{7 & 1 & \red{nil} & \red{4} & \red{20}}
\end{statetable} \ \\
\end{frame}

\begin{frame}[fragile]{Shadowing}
\begin{statetable}
{|c|c|c|c|c|}{PC & x & f & PC & x}{7 & 1 & nil & 4 & 20}
\end{statetable} \ \\

\begin{lstlisting}
x = 1

def f(x):
  return x * 2

print(f(10))
print(f(20))
\end{lstlisting}

\pause

\begin{statetable}
{|c|c|c|}{PC & x & f}{\red{8} & 1 & \red{40}}
\end{statetable} \ \\
\end{frame}


\begin{comment}

\SlideSubSection{Recursion}
\begin{slide}{
\item (Recursive) functions are all functions that call themselves in their bodies
\item This is based on the principle of induction and in general a very powerful technique
\item This leads to a compacter and often more easily correct representation
\begin{itemize}
\item Code is not easier to read, especially to the untrained eye
\end{itemize}
}\end{slide}

\begin{slide}{
\item Remember that calling a function creates a new instance of stack memory
\item Recursive functions do this a lot
\item Each recursive call has its own environment
}\end{slide}

\begin{frame}[fragile]{Recursion}
\begin{codewithblock}{\item How much is the distance between two numbers \texttt{n} and \texttt{m}? \pause \item As many as the units between them}
def distance(n,m):
  if n >= m:
    return 0
  else:
    return sum_between(n + 1, m) + 1
\end{codewithblock}
\end{frame}

\begin{frame}[fragile]{Recursion}
\begin{statetable}
{|c|}{PC}{7}
\end{statetable} \ \\

\begin{lstlisting}
def distance(n,m):
  if n >= m:
    return 0
  else:
    return sum_between(n + 1, m) + 1
    
print(distance(2, 4)))
\end{lstlisting}

\pause

\begin{statetable}
{|c|c|c|c|c|}{PC & distance & PC & n & m}{7 & \red{nil} & 2 & \red{2} & \red{4}}
\end{statetable} \ \\
\end{frame}

\begin{frame}[fragile]{Recursion}
\begin{statetable}
	{|c|c|c|c|c|}{PC & distance & PC & n & m}{7 & nil & 2 & 2 & 4}
\end{statetable} \ \\

\begin{lstlisting}
def distance(n,m):
  if n >= m:
    return 0
  else:
    return sum_between(n + 1, m) + 1

print(distance(2, 4)))
\end{lstlisting}

\pause

\begin{statetable}
	{|c|c|c|c|c|}{PC & distance & PC & n & m}{7 & nil & \red{5} & 2 & 4}
\end{statetable} \ \\
\end{frame}

\begin{frame}[fragile]{Recursion}
\begin{statetable}
	{|c|c|c|c|c|}{PC & distance & PC & n & m}{7 & nil & 5 & 2 & 4}
\end{statetable} \ \\

\begin{lstlisting}
def distance(n,m):
  if n >= m:
    return 0
  else:
    return sum_between(n + 1, m) + 1

print(distance(2, 4)))
\end{lstlisting}

\pause

\begin{statetable}
	{|c|c|c|c|c|c|c|c|c|}{PC & distance & PC & n & m & distance & PC & n & m}{7 & nil & 5 & 2 & 4 & \red{nil} & \red{2} & \red{3} & \red{4}}
\end{statetable} \ \\
\end{frame}

\begin{frame}[fragile]{Recursion}
\begin{statetable}
	{|c|c|c|c|c|c|c|c|c|}{PC & distance & PC & n & m & distance & PC & n & m}{7 & nil & 5 & 2 & 4 & nil & 2 & 3 & 4}
\end{statetable} \ \\

\begin{lstlisting}
def distance(n,m):
  if n >= m:
    return 0
  else:
    return sum_between(n + 1, m) + 1

print(distance(2, 4)))
\end{lstlisting}

\pause

\begin{statetable}
	{|c|c|c|c|c|c|c|c|c|}{PC & distance & PC & n & m & distance & PC & n & m}{7 & nil & 5 & 2 & 4 & nil & \red{5} & 3 & 4}
\end{statetable} \ \\
\end{frame}

\begin{frame}[fragile]{Recursion}
\begin{statetable}
	{|c|c|c|c|c|c|c|c|c|}{PC & distance & PC & n & m & distance & PC & n & m}{7 & nil & 5 & 2 & 4 & nil & 5 & 3 & 4}
\end{statetable} \ \\

\begin{lstlisting}
def distance(n,m):
  if n >= m:
    return 0
  else:
    return sum_between(n + 1, m) + 1

print(distance(2, 4)))
\end{lstlisting}

\pause

\begin{statetable}
	{|c|c|c|c|c|c|c|c|c|c|c|c|c|}{PC & distance & PC & n & m & distance & PC & n & m & distance & PC & n & m}{7 & nil & 5 & 2 & 4 & nil & 5 & 3 & 4 & \red{nil} & \red{2} & \red{4} & \red{4}}
\end{statetable} \ \\
\end{frame}

\begin{frame}[fragile]{Recursion}
\begin{statetable}
	{|c|c|c|c|c|c|c|c|c|c|c|c|c|}{PC & distance & PC & n & m & distance & PC & n & m & distance & PC & n & m}{7 & nil & 5 & 2 & 4 & nil & 5 & 3 & 4 & nil & 2 & 4 & 4}
\end{statetable} \ \\

\begin{lstlisting}
def distance(n,m):
  if n >= m:
    return 0
  else:
    return sum_between(n + 1, m) + 1

print(distance(2, 4)))
\end{lstlisting}

\pause

\begin{statetable}
	{|c|c|c|c|c|c|c|c|c|c|c|c|c|}{PC & distance & PC & n & m & distance & PC & n & m & distance & PC & n & m}{7 & nil & 5 & 2 & 4 & nil & 5 & 3 & 4 & nil & \red{3} & 4 & 4}
\end{statetable} \ \\
\end{frame}

\begin{frame}[fragile]{Recursion}
	\begin{statetable}
{|c|c|c|c|c|c|c|c|c|c|c|c|c|}{PC & distance & PC & n & m & distance & PC & n & m & distance & PC & n & m}{7 & nil & 5 & 2 & 4 & nil & 5 & 3 & 4 & nil & 3 & 4 & 4}
	\end{statetable} \ \\
	
	\begin{lstlisting}
def distance(n,m):
  if n >= m:
    return 0
  else:
    return sum_between(n + 1, m) + 1

print(distance(2, 4)))
	\end{lstlisting}
	
	\pause
	
	\begin{statetable}
		{|c|c|c|c|c|c|c|c|c|c|}{PC & distance & PC & n & m & distance & PC & n & m & distance }{7 & nil & 5 & 2 & 4 & nil & 5 & 3 & 4 & \red{0} }
	\end{statetable} \ \\
\end{frame}
\begin{frame}[fragile]{Recursion}
	\begin{statetable}
		{|c|c|c|c|c|c|c|c|c|c|}{PC & distance & PC & n & m & distance & PC & n & m & distance }{7 & nil & 5 & 2 & 4 & nil & 5 & 3 & 4 & 0 }
	\end{statetable} \ \\
	
	\begin{lstlisting}
def distance(n,m):
 if n >= m:
  return 0
 else:
  return sum_between(n + 1, m) + 1
	
print(distance(2, 4)))
	\end{lstlisting}
	
	\pause
	
	\begin{statetable}
		{|c|c|c|c|c|c|}{PC & distance & PC & n & m & distance }{7 & nil & 5 & 2 & 4 & \red{1} }
	\end{statetable} \ \\
\end{frame}

\begin{frame}[fragile]{Recursion}
	\begin{statetable}
		{|c|c|c|c|c|c|}{PC & distance & PC & n & m & distance }{7 & nil & 5 & 2 & 4 & 1 }
	\end{statetable} \ \\
	
	\begin{lstlisting}
def distance(n,m):
  if n >= m:
    return 0
  else:
    return sum_between(n + 1, m) + 1
	
print(distance(2, 4)))
	\end{lstlisting}
	
	\pause
	
	\begin{statetable}
		{|c|c|}{PC & distance }{7 & \red{2} }
	\end{statetable} \ \\
\end{frame}

\end{comment}

\SlideSection{Assignments}
\SlideSubSection{Build and test, on paper...}
\begin{slide}{
\item A function \texttt{add} that increments a given number by a fixed value:
\begin{itemize}
\item \texttt{add(1, 5) -> 6}
\end{itemize}
\item A function \texttt{isEven} that returns \texttt{True} if a given number is even, \texttt{False} otherwise:
\begin{itemize}
\item \texttt{isEven(6) -> True}
\end{itemize}
\item A function \texttt{sum\_between} that returns the sum of all integers between two given integer numbers:
\begin{itemize}
\item \texttt{sum\_between(2, 5) -> 14}
\end{itemize}
}\end{slide}

\SlideSection{Conclusion}
\SlideSubSection{Lecture topics}
\begin{slide}{
\item Often, user code needs to perform operations that are similar to each other
\item Through the mechanism of function definition, we can recycle code
\item Functions can encode algorithms in many way
\begin{itemize}
\item Simple code abstractions to avoid repetition
\end{itemize}
}\end{slide}

\begin{thankyou}
\end{thankyou}

\end{document}

\begin{slide}{
\item ...
}\end{slide}

\begin{frame}[fragile]
\begin{lstlisting}
...
\end{lstlisting}
\end{frame}
