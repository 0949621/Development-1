\documentclass{beamer}
\usetheme[hideothersubsections]{HRTheme}
\usepackage{beamerthemeHRTheme}
\usepackage{graphicx}
\usepackage[space]{grffile}
\usepackage{listings}
\lstset{language=C,
basicstyle=\ttfamily\footnotesize,
mathescape=true,
breaklines=true}
\usepackage[utf8]{inputenc}
\usepackage{color}

\title{Conditionals}

\author{The INFDEV Team @ HR}

\institute{Hogeschool Rotterdam \\ 
Rotterdam, Netherlands}

\date{}

\begin{document}
\maketitle

\SlideSection{Introduction}
\SlideSubSection{Lecture topics}
\begin{slide}{
\item Making choices
\item \texttt{if-then-else} statements
\item Reasoning about \texttt{if-then-else}
}\end{slide}

\SlideSection{Making choices}
\SlideSubSection{Making choices}
\begin{slide}{
\item Often need to \textit{make a choice}
\item Based on some \textit{condition}, we do \textit{something} rather than \textit{something else}
}\end{slide}

\begin{slide}{
\item If \textit{the sun is shining}
\item Then \textit{take a walk}
\item Otherwise \textit{go to work}
}\end{slide}

\begin{slide}{
\item If \textit{the engine is too warm and the RPM's are high enough}
\item Then \textit{reduce the RPM}
\item Otherwise \textit{do nothing}
}\end{slide}

\begin{slide}{
\item Of course conditions like this can be combined
\item This means that we can \textit{cascade} decisions
\item This is the building block of \textit{intelligent decisions} in our programs
}\end{slide}

\begin{slide}{
\item If \textit{the engine is too warm}
\item Then
\begin{itemize}
\item If \textit{the RPM's are high enough}
\item Then \textit{reduce the RPM}
\item Otherwise \textit{light up the temperature lamp}
\end{itemize}
\item Otherwise \textit{do nothing}
}\end{slide}

\SlideSection{Making decisions in Python}
\SlideSubSection{\texttt{if-then-else}}
\begin{slide}{
\item Python offers built-in facilities for decision-making
\item \texttt{if-then-else} statement
\item We can make decisions about which block of code is executed
}\end{slide}

\begin{slide}{
\item The general form is \texttt{if CONDITION: THEN-BLOCK else ELSE-BLOCK} ($if_{CTE}$)
\item If the condition is true, then we jump to the beginning of \texttt{THEN-BLOCK}, otherwise we jump to the beginning of \texttt{ELSE-BLOCK}
$$
\tiny
\left\{
\begin{matrix}
(PC,S) \overset{if_{CTE}}{\rightarrow} (firstLine(T),S) &when& (PC,S) \overset{C}{\rightarrow} TRUE & \\
(PC,S) \overset{if_{CTE}}{\rightarrow} (firstLine(E),S) &when& (PC,S) \overset{C}{\rightarrow} FALSE & \\
\end{matrix}
\right.
$$
}\end{slide}

\begin{slide}{
\item Python is \textit{indentation}-based
\item White-spaces go at the beginning of some lines
\item A more indented line is \textit{within} a less indented line above
}\end{slide}

\begin{slide}{
\item Indentation specifies where the \texttt{then-block} and the \texttt{else-block} begin and end
\item The general form of an \texttt{if-then-else} is thus:
\begin{itemize}
\item \texttt{if COND:}
\item \texttt{newline}
\item \textbf{indentation}
\item \texttt{code of then}
\item \textbf{de-indentation}
\item \texttt{else:}
\item \texttt{newline}
\item \textbf{indentation}
\item \texttt{code of else}
\item \textbf{de-indentation}
\end{itemize}
}\end{slide}

\begin{frame}[fragile]{A correct example}
\begin{lstlisting}[frame=shadowbox]
if temp > 350.0:
  if throttle > 2500:
    throttle = throttle - 1500
  else:
    warning = True
else:
  print("everything is OK")
\end{lstlisting}
\end{frame}

\begin{frame}[fragile]{An incorrect example}
\begin{lstlisting}[frame=shadowbox]
if temp > 350.0:
if throttle > 2500:
throttle = throttle - 1500
else:
warning = True
else:
print("everything is OK")
\end{lstlisting}
\end{frame}

\begin{slide}{
\item \texttt{if-then-else} statements eventually terminate
\item after the \texttt{then} (or \texttt{else}) block is finished, we jump to the first line right after the whole \texttt{if-then-else}
}\end{slide}

\begin{frame}[fragile]{After an \texttt{if-then-else}}
\begin{lstlisting}[frame=shadowbox]
if temp > 350.0:
  if throttle > 2500:
    throttle = throttle - 1500
  else:
    warning = True
else:
  print("everything is OK")
print(throttle, temp, warning)
\end{lstlisting}
\end{frame}

\begin{frame}[fragile]{After an \texttt{if-then-else}?}
Without indentation, this:
\begin{lstlisting}[frame=shadowbox,basicstyle=\ttfamily\tiny]
if temp > 350.0:
if throttle > 2500:
throttle = throttle - 1500
else:
warning = True
else:
print("everything is OK")
print(throttle, temp, warning)
\end{lstlisting}

would be indistinguishable from both:

\begin{tabular}{| c | c |}
\hline
\begin{lstlisting}[basicstyle=\ttfamily\tiny]
if temp > 350.0:
  if throttle > 2500:
    throttle = throttle - 1500
  else:
    warning = True
else:
  print("everything is OK")
  print(throttle, temp, warning)
\end{lstlisting}
&
\begin{lstlisting}[basicstyle=\ttfamily\tiny]
if temp > 350.0:
  if throttle > 2500:
    throttle = throttle - 1500
  else:
    warning = True
else:
  print("everything is OK")
print(throttle, temp, warning)
\end{lstlisting} \\
\hline
\end{tabular}
\end{frame}

\SlideSection{Reasoning about \texttt{if-then-else}}
\begin{slide}{
\item \texttt{if-then-else} effectively forks the code
\item Until run-time, we are not really sure what path the code will take
}\end{slide}

\begin{frame}[fragile]{Example \texttt{if}'s}
\begin{tabular}{| c | c |}
\hline
\begin{lstlisting}[basicstyle=\ttfamily\tiny]
x = 0
y = 0
z = 0
op = "none"
input = sys.stdin.readline()
if input == "*\n":
  x = int(sys.stdin.readline())
  y = int(sys.stdin.readline())
  op = "*"
else:
  if input == "+\n":
    x = int(sys.stdin.readline())
    y = int(sys.stdin.readline())
    op = "+"
  else:
    x = int(sys.stdin.readline())
    y = 2
    op = "*"
\end{lstlisting}
&
\begin{lstlisting}[basicstyle=\ttfamily\tiny]
if op == "+":
  z = x + y
else:
  if op == "*":
    z = x * y
  else:
    raise
print(str(x) + " " + op + " " + 
        str(y) + " is " + str(z))
\end{lstlisting} \\
\hline
\end{tabular}
\end{frame}

\begin{frame}[fragile]{Example \texttt{if}'s}
Which path will be taken?
\begin{lstlisting}[frame=shadowbox,basicstyle=\ttfamily\tiny]
x = 0
y = 0
z = 0
op = "none"
input = sys.stdin.readline()
if input == "*\n":
  x = int(sys.stdin.readline())
  y = int(sys.stdin.readline())
  op = "*"
else:
  if input == "+\n":
    x = int(sys.stdin.readline())
    y = int(sys.stdin.readline())
    op = "+"
  else:
    x = int(sys.stdin.readline())
    y = 2
    op = "*"
\end{lstlisting}

\pause

\ \\

\textbf{We do not know!}
\end{frame}

\begin{frame}[fragile]{Example \texttt{if}'s}
What values will \texttt{x, y, op, input} have?
\begin{lstlisting}[frame=shadowbox,basicstyle=\ttfamily\tiny]
x = 0
y = 0
z = 0
op = "none"
input = sys.stdin.readline()
if input == "*\n":
  x = int(sys.stdin.readline())
  y = int(sys.stdin.readline())
  op = "*"
else:
  if input == "+\n":
    x = int(sys.stdin.readline())
    y = int(sys.stdin.readline())
    op = "+"
  else:
    x = int(sys.stdin.readline())
    y = 2
    op = "*"
\end{lstlisting}

\pause

\ \\

\textbf{We do not know!}
\end{frame}

\begin{slide}{
\item The paths are influenced by the value of the \texttt{input} variable
\begin{itemize}
\item One path for \texttt{"*\ts n"}
\item Another for \texttt{"+\ts n"}
\item Another for all other possible values
\end{itemize}
\item We analyze our code based on all possible outcomes
}\end{slide}


\begin{frame}[fragile]{Example \texttt{if}'s}
\texttt{\center
\begin{tabular}{| l | l | l | l | l |}
\hline
x & y & z & op & input \\
\hline
0 & 0 & 0 & "none" & \red{"*\ts n"} \\
\hline
\end{tabular}
} \ \\ \ \\ \ \\

\begin{lstlisting}[frame=shadowbox,basicstyle=\ttfamily\tiny]
if input == "*\n":
  x = int(sys.stdin.readline())
  y = int(sys.stdin.readline())
  op = "*"
else:
  ...
\end{lstlisting}

\pause

\begin{tabular}{| l | l | l | l | l |}
\hline
x & y & z & op & input \\
\hline
\red{in2} & \red{in3} & 0 & \red{"*"} & "*\ts n" \\
\hline
\end{tabular}
\end{frame}

\begin{frame}[fragile]{Example \texttt{if}'s}
\texttt{\center
\begin{tabular}{| l | l | l | l | l |}
\hline
x & y & z & op & input \\
\hline
0 & 0 & 0 & "none" & \red{"+\ts n"} \\
\hline
\end{tabular}
} \ \\ \ \\ \ \\

\begin{lstlisting}[frame=shadowbox,basicstyle=\ttfamily\tiny]
if input == "*\n":
  ...
else:
  if input == "+\n":
    x = int(sys.stdin.readline())
    y = int(sys.stdin.readline())
    op = "+"
  else:
    ...
\end{lstlisting}

\pause

\begin{tabular}{| l | l | l | l | l |}
\hline
x & y & z & op & input \\
\hline
\red{in2} & \red{in3} & 0 & \red{"+"} & "+\ts n" \\
\hline
\end{tabular}
\end{frame}

\begin{frame}[fragile]{Example \texttt{if}'s}
\texttt{\center
\begin{tabular}{| l | l | l | l | l |}
\hline
x & y & z & op & input \\
\hline
0 & 0 & 0 & "none" & \red{"anything else"} \\
\hline
\end{tabular}
} \ \\ \ \\ \ \\

\begin{lstlisting}[frame=shadowbox,basicstyle=\ttfamily\tiny]
if input == "*\n":
  ...
else:
  if input == "+\n":
    ...
  else:
    x = int(sys.stdin.readline())
    y = 2
    op = "*"
\end{lstlisting}

\pause

\begin{tabular}{| l | l | l | l | l |}
\hline
x & y & z & op & input \\
\hline
\red{in2} & \red{2} & 0 & \red{"*"} & "anything else" \\
\hline
\end{tabular}
\end{frame}

\begin{frame}[fragile]{Example \texttt{if}'s}
We can now merge the various possible outcomes (ignoring \texttt{input} as we do not use it anymore):

\begin{tabular}{| l | l | l | l | l |}
\hline
x & y & z & op & input \\
\hline
in2 & in3 & 0 & "*" & "*\ts n" \\
\hline
in2 & in3 & 0 & "+" & "+\ts n" \\
\hline
in2 & 2 & 0 & "*" & "anything else" \\
\hline
\end{tabular}

\pause

\begin{tabular}{| l | l | l | l |}
\hline
x & y & z & op \\
\hline
\red{in2} & \red{in3 $\vee$ 2} & 0 & \red{"*" $\vee$ "+"} \\
\hline
\end{tabular}
\end{frame}

\begin{frame}[fragile]{Example \texttt{if}'s}
\texttt{\center
\begin{tabular}{| l | l | l | l |}
\hline
x & y & z & op \\
\hline
in2 & in3 $\vee$ 2 & 0 & "*" $\vee$ "+" \\
\hline
\end{tabular}
} \ \\ \ \\ \ \\

\begin{lstlisting}[frame=shadowbox,basicstyle=\ttfamily\tiny]
if op == "+":
  z = x + y
else:
  if op == "*":
    z = x * y
  else:
    raise
\end{lstlisting}

\pause

\begin{tabular}{| l | l | l | l |}
\hline
x & y & z & op \\
\hline
in2 & in3 $\vee$ 2 & \red{in2+in3 $\vee$ in2$\times$in3 $\vee$ in2$\times$2} & "*" $\vee$ "+" \\
\hline
\end{tabular}
\end{frame}

\SlideSubSection{Exponential explosion of potential control-paths}
\begin{slide}{
\item \texttt{in2+in3 $\vee$ in2$\times$in3 $\vee$ in2$\times$2} is a long formula
\item It is simply saying that there are three possible outcomes:
\begin{itemize}
\item One outcome is \texttt{in2+in3}
\item One outcome is \texttt{in2$\times$in3}
\item One outcome is \texttt{in2$\times$2}
\end{itemize}
}\end{slide}

\begin{slide}{
\item The more sequential conditionals, the more possible resulting execution paths
\item But \textit{how many?}
}\end{slide}

\begin{frame}[fragile]{Sequential \texttt{if}'s}
How many \texttt{if}'s? \\ 
How many execution paths? \\ \ \\

\begin{lstlisting}[frame=shadowbox,basicstyle=\ttfamily\tiny]
if C1:
  A1
else:
  B1
\end{lstlisting}

\pause

\textbf{1} \texttt{if} \\
\textbf{2} execution paths \\
\end{frame}

\begin{frame}[fragile]{Sequential \texttt{if}'s}
\begin{lstlisting}[frame=shadowbox,basicstyle=\ttfamily\tiny]
if C1:
  A1
else:
  B1

if C2:
  A2
else:
  B2
\end{lstlisting}

\pause

\textbf{2} \texttt{if's} \\
\textbf{4} execution paths \\
\end{frame}

\begin{frame}[fragile]{Sequential \texttt{if}'s}
\begin{lstlisting}[frame=shadowbox,basicstyle=\ttfamily\tiny]
if C1:
  A1
else:
  B1

if C2:
  A2
else:
  B2

if C3:
  A3
else:
  B3
\end{lstlisting}

\pause

\textbf{3} \texttt{if's} \\
\textbf{8} execution paths \\
\end{frame}

\begin{slide}{
\item In general, for \texttt{n} \texttt{if}'s
\item $2^n$ possible execution paths 
}\end{slide}

\begin{slide}{
\item Each path can alter the state in a different way
\item After an \texttt{if} with 8 possible paths
\begin{itemize}
\item We have 8 possible resulting states
\item Variables can be one of possible 8 different values
\end{itemize}
}\end{slide}

\begin{slide}{
\item The more \texttt{if}'s
\item The more complex its conditions
\item \textit{The harder it is to reason about your program!}
}\end{slide}

\SlideSection{Using \texttt{if}'s}
\SlideSubSection{Rules of thumb}
\begin{slide}{
\item Logical, short condition
\item Good: \texttt{(temp > 350 \& throttle > 2500)}
\item Bad: \texttt{(temp > 350 \& throttle > 2500 \& op == "+")}
}\end{slide}

\begin{slide}{
\item Few levels of nesting
\item Good: between one and three
\item Bad: beyond three
}\end{slide}

\begin{slide}{
\item Semantically connected \texttt{then} and \texttt{else}
\item Good: both \texttt{then} and \texttt{else} perform similar operations on the same variables
\item Bad: \texttt{then} and \texttt{else} perform unrelated operations or on different variables
}\end{slide}

\begin{frame}[fragile]{A disastrous example}
\begin{lstlisting}[frame=shadowbox,basicstyle=\ttfamily\tiny]
if (temp > 350 & throttle > 2500) | op == "+":
  if op == "+":
    z = x + y
  else:
    z = x * y
    throttle = throttle - 1000
else:
  if op == "*":
    z = x * y
\end{lstlisting}

What went wrong? \\

\pause

\begin{itemize}
\item The condition is very hard to reason about
\item The condition involves unrelated quantities
\item The various \texttt{then}'s and \texttt{else}'s are partially unrelated
\item There is repetition
\end{itemize}
\end{frame}

\begin{frame}[fragile]{Bringing order}
\begin{lstlisting}[frame=shadowbox,basicstyle=\ttfamily\tiny]
if temp > 350 & throttle > 2500:
  throttle = throttle - 1000

if op == "+":
  z = x + y
else:
  z = x * y
\end{lstlisting}

What went right? \\

\pause

\begin{itemize}
\item The conditions are simple to reason about
\item The conditions are all tight (no unrelated variables)
\item The various \texttt{then}'s and \texttt{else}'s are all strongly related
\item Separate \texttt{if}'s for separate tasks
\item There is no repetition
\end{itemize}
\end{frame}

\SlideSubSection{The value of reasoning}
\begin{slide}{
\item \textbf{Always keep in mind:}
\item You have the power to make your own life a living Hell...
\pause
\item ...unless you reason first and then structure code logically
}\end{slide}

\begin{frame}{This is it!}
\center
\fontsize{18pt}{7.2}\selectfont
The best of luck, and thanks for the attention!
\end{frame}

\end{document}

\begin{slide}{
\item ...
}\end{slide}

\begin{frame}[fragile]
\begin{lstlisting}
...
\end{lstlisting}
\end{frame}
