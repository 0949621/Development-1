\documentclass{beamer}
\usetheme[hideothersubsections]{HRTheme}
\usepackage{beamerthemeHRTheme}
\usepackage{graphicx}
\usepackage[space]{grffile}
\usepackage{listings}
\lstset{language=C,
basicstyle=\ttfamily\footnotesize,
mathescape=true,
breaklines=true}
\usepackage[utf8]{inputenc}
\usepackage{color}

\title{Hello Python!}

\author{The INFDEV Team @ HR}

\institute{Hogeschool Rotterdam \\ 
Rotterdam, Netherlands}

\date{}

\begin{document}
\maketitle

\SlideSection{Introduction}
\SlideSubSection{Lecture topics}
\begin{slide}{
\item We introduce Python
\item We bridge what we have seen in the previous lecture with actual Python elements
}\end{slide}

\SlideSubSection{Why many programming languages?}
\begin{slide}{
\item Low-level vs high-level
\item Statically-typed vs dynamically-typed
\item Compiled vs interpreted
\item Imperative vs functional vs logic vs declarative vs object-oriented
\item Safe vs unsafe
\item Fast vs slow
\item ...
}\end{slide}

\begin{slide}{
\item The set of all problems is a complex, fractal-looking shape
\item The programming language we choose shifts our focus on these problems
\item Some become more visible and obvious to solve...
\pause
\item ...others become hidden, obstructed, or harder to solve
}\end{slide}

\begin{slide}{
\item Not all languages are equal
\item There is improvement and an ordering
\begin{itemize}
\item For low-level programming C is in most cases better than assembly
\item For data transformation SQL is in most cases better than Java
\item For algorithmic work on trees F\# is in most cases better than C\#
\end{itemize}
}\end{slide}

\begin{slide}{
\item Not all languages are comparable
\item There are perfectly valid differences in balance and features
\begin{itemize}
\item Most languages are better than assembly in most scenarios
\item For data transformation SQL is as good as F\# on algorithmic work on trees
\end{itemize}
}\end{slide}

\SlideSubSection{Early programming languages}
\begin{slide}{
\item Analytical Engine/Difference Engine: hypothetical mechanical computers (1840's, Charles Babbage and Ada Lovelace)
\item Assembly language: programming close to the machine (1940's)
\item Fortran, ALGOL, and COBOL: various forms of imperative programming (1950's)
\item LISP: functional and meta-programming (1950's, still in use)
\item Simula: object-oriented programming (1950's)
\item C: high-level low-level programming (1970's, still in use)
\item Smalltalk: everything-is-an-object programming (1970's)
}\end{slide}

\SlideSubSection{Early programming languages}
\begin{slide}{
\item Prolog: logic programming (1970's)
\item ML: statically typed, polymorphic functional programming (1970's, still in use)
\item SQL: query language (1970's, still in use)
}\end{slide}

\SlideSubSection{1980's}
\begin{slide}{
\item C++: C with classes (still in use)
\item Matlab and Mathematica: mathematics and simulations (still in use)
\item Erlang: concurrency and telecommunications (still in use)
}\end{slide}

\SlideSubSection{1990's: the Internet Age}
\begin{slide}{
\item Haskell: functional programming (still in use)
\item Python, Ruby, Lua: concise, dynamic programming (still in use)
\item JavaScript: webpage dynamics (still in use)
\item Java: objects and portability (still in use)
}\end{slide}

\SlideSubSection{2000's: the Modern Age}
\begin{slide}{
\item C\#: objects and portability (still in use)
\item F\# and Scala: hybrid, functional-first programming and portability (still in use)
\item Go and Swift: native, safe development (getting traction?)
}\end{slide}


\SlideSection{The Python Programming Language}
\SlideSubSection{The Python Zen}
\begin{slide}{
\item Beautiful is better than ugly
\item Explicit is better than implicit
\item Simple is better than complex
\item Complex is better than complicated
\item Readability counts
}\end{slide}

\SlideSubSection{Python introduction}
\begin{slide}{
\item General-purpose language
\item High-level
\item Concise on purpose
\item Dynamically typed
\item Hybrid paradigm, imperative/procedural first
}\end{slide}

\SlideSubSection{Why Python?}
\begin{slide}{
\item Used a lot as a beginning languages in higher education
\item Adequate for expressing the basics of computational thinking
\item High signal to noise ratio of syntax
}\end{slide}

\SlideSection{Python basic syntax and semantics}
\SlideSubSection{Variables}
\begin{slide}{
\item Variables are not declared
\item Just initialize and subsequently use
}\end{slide}

\SlideSubSection{Variable names}
\begin{slide}{
\item Variables may begin with any letter or the \texttt{\_} sign
\item Followed by any sequence of letters, numbers, and the \texttt{\_}
}\end{slide}

\begin{frame}[fragile]
\begin{lstlisting}
x
y
_x
customer_name
_x1
_x1_customer
\end{lstlisting}
\end{frame}

\SlideSubSection{Numbers}
\begin{slide}{
\item Python supports integers and other sorts of numbers
\item Any sequence of numeric characters (we call it an integer \textit{literal}) is a number
}\end{slide}

\begin{frame}[fragile]
\begin{lstlisting}
100
0
-1
79228162514264337593543950336L
\end{lstlisting}
\end{frame}

\SlideSubSection{Variable assignment}
\begin{slide}{
\item We can assign a value to a variable
\item \texttt{variableName = expression}
\item What does this do to the memory of the program? \textbf{Discuss.}
\pause
\item \textbf{If the variable did not exist, then we add it to memory}
\item If the variable existed, then we change its value in memory 
$$
\small
\left\{
\begin{array}{ll}
(PC,S) \overset{x = e}{\rightarrow} (PC+1,S[x \mapsto e]) &\mathrm{when}~x \in S \\
(PC,S) \overset{x = e}{\rightarrow} (PC+1,S\cup\{x \mapsto e\}) &\mathrm{otherwise}\\
\end{array}
\right.
$$
}\end{slide}

\begin{frame}[fragile]
\texttt{\center
\begin{tabular}{| l |}
\hline
PC \\
\hline
1 \\
\hline
\end{tabular}
} \ \\ \ \\ \ \\

\begin{lstlisting}[frame=shadowbox,numbers=left]
x = 100
y = 200
z = 50
\end{lstlisting}

what changes while running the current instruction? \textbf{Try to guess and discuss!}
\end{frame}

\begin{frame}[fragile]
\texttt{\center
\begin{tabular}{| l |}
\hline
PC \\
\hline
1 \\
\hline
\end{tabular}
} \ \\ \ \\ \ \\

\begin{lstlisting}[frame=shadowbox,numbers=left]
x = 100
y = 200
z = 50
\end{lstlisting}

\pause

\begin{tabular}{| l | l |}
\hline
PC & \red{x} \\
\hline
\red{2} & \red{100} \\
\hline
\end{tabular}
\end{frame}

\begin{frame}[fragile]
\texttt{\center
\begin{tabular}{| l | l |}
\hline
PC & x \\
\hline
2 & 100 \\
\hline
\end{tabular}
} \ \\ \ \\ \ \\

\begin{lstlisting}[frame=shadowbox,numbers=left]
x = 100
y = 200
z = 50
\end{lstlisting}

\pause

\begin{tabular}{| l | l | l |}
\hline
PC & x & \red{y} \\
\hline
\red 3 & 100 & \red{200} \\
\hline
\end{tabular}
\end{frame}

\begin{frame}[fragile]
\texttt{\center
\begin{tabular}{| l | l | l |}
\hline
PC & x & y \\
\hline
3 & 100 & 200 \\
\hline
\end{tabular}
} \ \\ \ \\ \ \\

\begin{lstlisting}[frame=shadowbox,numbers=left]
x = 100
y = 200
z = 50
\end{lstlisting}

\pause

\begin{tabular}{| l | l | l | l |}
\hline
PC & x & y & \red{z} \\
\hline
\red 4 & 100 & 200 & \red{50} \\
\hline
\end{tabular}
\end{frame}

\begin{slide}{
\item We can assign a value to a variable
\item \texttt{variableName = expression}
\item What does this do to the memory of the program? \textbf{Discuss.}
\pause
\item If the variable did not exist, then we add it to memory
\item \textbf{If the variable existed, then we change its value in memory} 
$$
\small
\left\{
\begin{array}{ll}
(PC,S) \overset{x = e}{\rightarrow} (PC+1,S[x \mapsto e]) &\mathrm{when}~x \in S \\
(PC,S) \overset{x = e}{\rightarrow} (PC+1,S\cup\{x \mapsto e\}) &\mathrm{otherwise}\\
\end{array}
\right.
$$
}\end{slide}

\begin{frame}[fragile]
\texttt{\center
\begin{tabular}{| l | l | l | l |}
\hline
PC & x & y & z \\
\hline
1 & 0 & -1 & 5 \\
\hline
\end{tabular}
} \ \\ \ \\ \ \\

\begin{lstlisting}[frame=shadowbox,numbers=left]
x = 100
y = 200
z = 50
\end{lstlisting}

what changes while running the current instruction? \textbf{Try to guess and discuss!}
\end{frame}

\begin{frame}[fragile]
\texttt{\center
\begin{tabular}{| l | l | l | l |}
\hline
PC & x & y & z \\
\hline
1 & 0 & -1 & 5 \\
\hline
\end{tabular}
} \ \\ \ \\ \ \\

\begin{lstlisting}[frame=shadowbox,numbers=left]
x = 100
y = 200
z = 50
\end{lstlisting}

\pause

\texttt{\center
\begin{tabular}{| l | l | l | l |}
\hline
PC & x & y & z \\
\hline
\red{2} & \red{100} & -1 & 5 \\
\hline
\end{tabular}
}
\end{frame}

\begin{frame}[fragile]
\texttt{\center
\begin{tabular}{| l | l | l | l |}
\hline
PC & x & y & z \\
\hline
2 & 100 & -1 & 5 \\
\hline
\end{tabular}
} \ \\ \ \\ \ \\

\begin{lstlisting}[frame=shadowbox,numbers=left]
x = 100
y = 200
z = 50
\end{lstlisting}

\pause

\texttt{\center
\begin{tabular}{| l | l | l | l |}
\hline
PC & x & y & z \\
\hline
\red{3} & 100 & \red{200} & 5 \\
\hline
\end{tabular}
}
\end{frame}

\begin{frame}[fragile]
\texttt{\center
\begin{tabular}{| l | l | l | l |}
\hline
PC & x & y & z \\
\hline
3 & 100 & 200 & 5 \\
\hline
\end{tabular}
} \ \\ \ \\ \ \\

\begin{lstlisting}[frame=shadowbox,numbers=left]
x = 100
y = 200
z = 50
\end{lstlisting}

\pause

\texttt{\center
\begin{tabular}{| l | l | l | l |}
\hline
PC & x & y & z \\
\hline
\red{4} & 100 & 200 & \red{50} \\
\hline
\end{tabular}
}
\end{frame}

\begin{frame}{This is it!}
\center
\fontsize{18pt}{7.2}\selectfont
The best of luck, and thanks for the attention!
\end{frame}

\end{document}

\begin{slide}{
\item ...
}\end{slide}

\begin{frame}[fragile]
\begin{lstlisting}
...
\end{lstlisting}
\end{frame}
