\documentclass{beamer}
\usetheme[hideothersubsections]{HRTheme}
\usepackage{beamerthemeHRTheme}
\usepackage{graphicx}
\usepackage[space]{grffile}
\usepackage{listings}
\lstset{language=C,
basicstyle=\ttfamily\footnotesize,
mathescape=true,
breaklines=true}
\usepackage[utf8]{inputenc}
\usepackage{color}


\title{Types}

\author{The DEV-team @ HR}

\institute{Hogeschool Rotterdam \\ 
Rotterdam, Netherlands}

\date{}

\begin{document}
\maketitle

\SlideSection{Introduction}
\SlideSubSection{Lecture topics}
\begin{slide}{
\item We introduce the Python type system
\item Numbers
\item Boolean values
\item Arithmetic and boolean expressions
}\end{slide}

\SlideSection{Python type system basics}
\SlideSubSection{Introduction}
\begin{slide}{
\item Is everything an integer number?
\item Yes and no
}\end{slide}

\SlideSubSection{Everything is an integer number}
\begin{slide}{
\item For the CPU everything is a string of bits
\item So yes, everything is (\textit{almost}\footnote{also floats are recognized by the CPU}) an integer number
\item Complex data structures like a GUI, a 3D model, a picture, etc. are made up of collections of numbers
}\end{slide}

\begin{slide}{
\item Low-level languages expose this view
\item Everything is encoded with numbers
\item It can become quite messy
}\end{slide}

\SlideSubSection{Not everything is an integer number}
\begin{slide}{
\item For the programmer, there exist different kinds of values
\item So common and useful that Python offers them out of the box
\item Even if the CPU does not manipulate them directly
}\end{slide}

\SlideSubSection{Kinds of values}
\begin{slide}{
\item Python has a \textbf{type system}
\item Variables have different \textbf{data types}, often shortened to \textbf{types}
\begin{itemize}
\item \texttt{Integer} numbers
\item Rational (\texttt{floating point}) numbers
\item \texttt{Boolean} truth values
\item \texttt{Strings} of text
\end{itemize}
}\end{slide}

\SlideSubSection{Integers}
\begin{slide}{
\item Numbers without dot\footnote{comma in Dutch}
\begin{itemize}
\item \texttt{0}
\item \texttt{100} 
\item \texttt{-500}
\end{itemize}
}\end{slide}

\begin{slide}{
\item Typical arithmetic operations on numbers (\textbf{not in Python 3})
\begin{itemize}
\item \texttt{3 + 5 = 8}
\item \texttt{5 / 2 = 2}
\item \texttt{40 * 5 = 200}
\end{itemize} 
}\end{slide}

\SlideSubSection{Floating points}
\begin{slide}{
\item Numbers with dot\footnote{comma in Dutch}
\begin{itemize}
\item \texttt{0.0}
\item \texttt{2.5}
\item \texttt{10.0e3}
\item \texttt{3.1e-5}
\item \texttt{-.1e-5}
\end{itemize}
}\end{slide}

\SlideSubSection{The scientific notation}
\begin{slide}{
\item \texttt{0.00001} is annoying to write
\item we can write \texttt{1.e-4} instead
\item the sign \texttt{e-N} means \textit{add N zeros right after the dot}
}\end{slide}

\begin{slide}{
\item \texttt{1000000.0} is annoying to write
\item we can write \texttt{1.e6} instead
\item the sign \texttt{eN} means \textit{add N zeros right before the dot}
}\end{slide}

\SlideSubSection{Floating points}
\begin{slide}{
\item Typical arithmetic operations on numbers
\begin{itemize}
\item \texttt{5.0 / 2.0 = ?} 
\item \texttt{10.0e3 / 0.1 = ?}
\item \texttt{3.1e-5 + 1.0e5 = ?}
\end{itemize}
\item \textbf{Can you guess the results?}
\pause
\begin{itemize}
\item \texttt{5.0 / 2.0 = 2.5}
\item \texttt{10.0e3 / 0.1 = 10.0e4}
\item \texttt{3.1e-5 + 1.0e5 = 100000.000031}
\end{itemize} 
}\end{slide}

\SlideSubSection{Conversion to and from floating point}
\begin{slide}{
\item Integers can be converted to floating points with \texttt{float(n)}
\item Floating points can be converted to integers with \texttt{int(n)}
}\end{slide}

\begin{slide}{
\item Given the following expressions:
\begin{itemize}
\item \texttt{int(2.5) = ?}
\item \texttt{float(3) = ?}
\end{itemize}
\item \textbf{Can you guess the results?}
\pause
\begin{itemize}
\item \texttt{int(2.5) = 2}
\item \texttt{float(3) = 3.0}
\end{itemize}
}\end{slide}

\begin{slide}{
\item Floating points can lose their decimal values
\item They stay float's, but always end in \texttt{.0}
\item \texttt{math.floor(n)} truncates the tail
\item \texttt{math.ceil(n)} fills the tail and increases to the next unit
}\end{slide}

\begin{slide}{
\item Given the following expressions:
\begin{itemize}
\item \texttt{floor(2.5) = ?} 
\item \texttt{ceil(2.5) = ?}
\end{itemize}
\item \textbf{Can you guess the results?}
\pause
\begin{itemize}
\item \texttt{floor(2.5) = 2.0}
\item \texttt{ceil(2.5) = 3.0}
\end{itemize}
}\end{slide}

\begin{slide}{
\item Some conversions happen automatically
\item Python operations try to preserve information
\item \texttt{5 / 2.0 = 2.5}, and \texttt{5} is converted to \texttt{5.0} right before the division
}\end{slide}

\SlideSubSection{Python 3 integer division}
\begin{slide}{
\item The new version of Python has a new integer division: it always converts to float
\item It is \textbf{very different} from most other programming languages
\item \texttt{5 / 2 = 2.5}
}\end{slide}

\SlideSubSection{Python 3 integer division}
\begin{slide}{
\item Traditional integer division is now ``\texttt{\textbf{//}}"
\item \texttt{5 // 2 = 2}
}\end{slide}

\SlideSubSection{Boolean values}
\begin{slide}{
\item Truth values
\item \texttt{True}, \texttt{False}
\item ``Answers to yes/no questions''
}\end{slide}

\begin{slide}{
\item Logical operators on truth values
\item Compose the asnwers to multiple questions
\item Both questions in parallel:
\begin{itemize}
\item Do you like chocolate? Yes.
\item Do you like vanilla? Yes.
\item \textbf{Do you like chocolate and vanilla? Yes.}
\end{itemize}
\item Both questions concurrently:
\begin{itemize}
\item Do you like chocolate? Yes.
\item Do you like vanilla? No.
\item \textbf{Do you like chocolate or vanilla? Yes.}
\end{itemize}
\item Turn questions around:
\begin{itemize}
\item Do you like chocolate? Yes.
\item \textbf{Do you dislike chocolate? No.}
\end{itemize}
}\end{slide}

\begin{slide}{
\item Logical operators take one or two input
\item This means that we have no more than four possible combinations of input values
\item Since the inputs are so few, we can enumerate all combinations
\item This is done with a \textbf{truth table}
}\end{slide}

\begin{slide}{
\item Truth tables enumerate all input values and the result of their operator
\begin{tabular}{c c c}
A & B & ( A $\odot$ B ) \\
\hline 
$True$ & $True$ &  $\dots$ \\
$True$ & $False$ &  $\dots$ \\
$False$ & $True$ &  $\dots$ \\
$False$ & $False$ &  $\dots$ \\
\end{tabular}
}\end{slide}

\begin{slide}{
\item Logical operators on truth values
\begin{itemize}
\item \texttt{\&} for \texttt{and}
\\
%NOTE: requires \usepackage{color}
\begin{tabular}{@{ }c@{ }@{ }c | c@{}@{ }c@{ }@{ }c@{ }@{ }c@{ }@{}c@{ }}
A & B & ( A $\&$ B ) & \\
\hline 
$True$ & $True$ &  \textcolor{red}{$True$} \\
$True$ & $False$ &  \textcolor{red}{$False$} \\
$False$ & $True$ &  \textcolor{red}{$False$} \\
$False$ & $False$ &  \textcolor{red}{$False$} \\
\end{tabular}
\end{itemize}
}\end{slide}

\begin{slide}{
\item Logical operators on truth values
\begin{itemize}
\item \texttt{|} for \texttt{or}
%NOTE: requires \usepackage{color}
\\
\begin{tabular}{@{ }c@{ }@{ }c | c@{}@{ }c@{ }@{ }c@{ }@{ }c@{ }@{}c@{ }}
A & B & ( & A & $|$ & B & )\\
\hline 
$True$ & $True$ &  & $True$ & \textcolor{red}{$True$} & $True$ & \\
$True$ & $False$ &  & $True$ & \textcolor{red}{$True$} & $False$ & \\
$False$ & $True$ &  & $False$ & \textcolor{red}{$True$} & $True$ & \\
$False$ & $False$ &  & $False$ & \textcolor{red}{$False$} & $Fasle$ & \\
\end{tabular}
\end{itemize}
}\end{slide}

\begin{slide}{
\item Logical operators on truth values
\begin{itemize}
\item \texttt{not}
\\
%NOTE: requires \usepackage{color}
\begin{tabular}{@{ }c | c@{ }@{ }c}
A & $not$ & A\\
\hline 
$True$ & \textcolor{red}{$False$} & $True$\\
$False$ & \textcolor{red}{$True$} & $False$\\
\end{tabular}
\end{itemize}
}\end{slide}

\begin{slide}{
\item Comparison operators on numeric values
\begin{itemize}
\item \texttt{>}
\item \texttt{<}
\item \texttt{==}
\item \texttt{>=}
\item \texttt{<=}
\end{itemize}
}\end{slide}

\begin{slide}{
\item Given the following expressions:
\begin{itemize}
\item \texttt{5.0 > 2.0 = ?}
\item \texttt{(3 > 4) | (5 == (3 + 2)) = ?}
\item \texttt{True \& False = ?}
\end{itemize}
\item \textbf{Can you guess the results?}
\pause
\begin{itemize}
\item \texttt{5.0 > 2.0 = True}
\item \texttt{(3 > 4) | (5 == (3 + 2)) = True}
\item \texttt{True \& False = False}
\end{itemize}
}\end{slide}

\SlideSubSection{String values}
\begin{slide}{
\item Text
\item \texttt{"Hello!"}, \texttt{"Hello world!"}, \texttt{""}, ...
}\end{slide}

\begin{slide}{
\item String literals are sequences of characters, on a single line, between double \texttt{"} or single \texttt{'} quotes 
\item Some characters do not fit this description
\item We need special markings for such characters
\item These special markings are called \textit{escape characters}
}\end{slide}

\begin{slide}{
\small
\item \texttt{\textbackslash '} for single quote
\item \texttt{\textbackslash "} for double quote
\item \texttt{\textbackslash a} for ASCII Bell (BEL)  
\item \texttt{\textbackslash b} for ASCII Backspace (BS)  
\item \texttt{\textbackslash f} for ASCII Formfeed (FF)  
\item \texttt{\textbackslash n} for ASCII Linefeed (LF)  
\item \texttt{\textbackslash r} for ASCII Carriage Return (CR)  
\item \texttt{\textbackslash t} for ASCII Horizontal Tab (TAB)  
\item \texttt{\textbackslash v} for ASCII Vertical Tab (VT) 
}\end{slide}

\begin{slide}{
\item \texttt{"Hello\ts n world"} is a string on two lines
\item \texttt{"Hello\ts n world\ts n of Python"} is a string on three lines
\item ...
}\end{slide}

\begin{slide}{
\item The most common operator is string concatenation
\item \texttt{"Hello" + "\ts n" + "world" + "\ts n" + "on" + "\ts n" + "different" + "\ts n" + "lines"}
}\end{slide}

% common string functions
% samples of strings and escape characters

\SlideSection{Type restrictions}
\SlideSubSection{Operations, types, and restrictions}
\begin{slide}{
\item Not all operations are allowed on all possible variable types
\begin{itemize}
\item Some operations are allowed (integer addition)
\item Some operations are not allowed (string division)
\item Some operations change meaning (addition of integers versus concatenation of strings)
\end{itemize}
}\end{slide}

\begin{slide}{
\item Examples of allowed operators
\begin{itemize}
\item Addition, subtraction, division, multiplication, etc. between numbers
\item Concatenation between strings
\item Multiplication of strings and integers
\item Arithmetic comparison between numbers or strings
\item Conjunction, disjunction, negation\footnote{\texttt{and}, \texttt{or}, \texttt{not}} between booleans
\item Treating integers as booleans (\texttt{1=True}, \texttt{0=False})
\item Treating strings as booleans (anything else\texttt{=True}, \texttt{""=False})
\end{itemize}
}\end{slide}

\begin{slide}{
\item Examples of not-allowed operators
\begin{itemize}
\item Most arithmetic operations on strings and non-strings (\texttt{"Hello" + True})
\item Most boolean operations on strings and non-strings (\texttt{"Hello" \& True})
\end{itemize}
}\end{slide}

\begin{frame}[fragile]{Type errors}
Not-allowed operators generate \textit{type errors}

\begin{lstlisting}[frame=shadowbox]
Traceback (most recent call last):
  File "C:\Users\Giuseppe\Desktop\DEV I samples\DEV_I_samples.py", line 8, in <module>
    print("Oh noes, a bug!" + 4)
TypeError: cannot concatenate 'str' and 'int' objects
\end{lstlisting}
\end{frame}

\begin{slide}{
\item Variables may change type in Python
\item An integer variable becomes later on a string variable
\item This is allowed, but dangerous
\item A variable should never lose reasonable meaning
\item Many type errors stem from \textit{changes in meaning}, connected with \textit{changes in type} of a variable
}\end{slide}


\SlideSection{Operator precedence}
\SlideSubSection{Introduction}
\begin{slide}{
\item Multiple operators in a single expression are ambiguous
\item For example: \texttt{not True | True}
\begin{itemize}
\item \texttt{(not True) | True = False | True = True}
\item \texttt{not (True | True) = not True = False}
\end{itemize}
}\end{slide}

\begin{slide}{
\item Python defines which operators are evaluated first, and which later
\item Removes ambiguity
\item Makes parentheses not required
\begin{itemize}
\item Still, might remain better for readability
\end{itemize}
}\end{slide}

\begin{slide}{
\item From lowest precedence (least binding) to highest precedence (most binding) 
\pause
\item Some operators share the same precedence 
\begin{itemize}
\item $+$, $-$
\item $*$, $/$
\end{itemize}
\pause
\item Unless the syntax is explicitly given (example by mean of parenthesis)
\pause
\item A complete table of precedence can be found on \url{https://docs.python.org/2/reference/expressions.html\#operator-precedence}
}\end{slide}

\begin{slide}{
\item Example: integer operations in Python like $*$ and $/$ have higher precedence than $+$ and $-$
\item $1 + 4 * 2 = 9$
\pause
\item Use parenthesis to group expressions 
\item $(1 + 4) * 2 = 10$  
}\end{slide}

\begin{slide}{
\item Given the following expressions what are the results:
\begin{itemize}
\item (20 + 10) * 15 / 5 = ?
\item ((20 + 10) * 15) / 5 = ?
\item 20 + (10 * 15) / 5 = ?
\end{itemize}
\pause 
\item Results: 
\begin{itemize}
\item (20 + 10) * 15 / 5 = 90
\item ((20 + 10) * 15) / 5 = 90
\item 20 + (10 * 15) / 5 = 50
\end{itemize}
}\end{slide}


\SlideSection{Assignment}
\SlideSubSection{Instructions}
\begin{slide}{
\item Split into four groups.
\item Use the data types you saw in this lesson to model an RPG character in a Python program.
\item Example: health, team color, ...
\item Make sure the program runs without errors.
\item Draw on a sheet what the soldier should look like.
\item Hand over the code to another group and make them draw the soldier.
\item If the pictures are the same then you have succeeded, otherwise adjust your code.
}\end{slide}


\SlideSubSection{Hand-in}
\begin{slide}{
\item Write your names and student numbers on your sheets
\item Hand them in
\item \textit{They may be used at your oral check} in the form of questions such as ``how would you rewrite this after the course''
}\end{slide}


\begin{frame}{This is it!}
\center
\fontsize{18pt}{7.2}\selectfont
The best of luck, and thanks for the attention!
\end{frame}

\end{document}

\begin{slide}{
\item ...
}\end{slide}

\begin{frame}[fragile]
\begin{lstlisting}
...
\end{lstlisting}
\end{frame}


