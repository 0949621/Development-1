\documentclass{beamer}

\usetheme[hideothersubsections]{HRTheme}
\usepackage{beamerthemeHRTheme}

\usepackage{graphicx}
\usepackage[space]{grffile}
\usepackage{listings}
\lstset{language=C,
basicstyle=\ttfamily\footnotesize,
mathescape=true,
showspaces=false,
breaklines=true}
\usepackage[utf8]{inputenc}

\title{The logical model of computation}

\author{The INFDEV Team @ HR}

\institute{Hogeschool Rotterdam \\ 
Rotterdam, Netherlands}

\date{}

\begin{document}
\maketitle

\SlideSection{Introduction}
\SlideSubSection{Course topics}
\begin{slide}{
\item This course is about \textit{basic programming concepts} (DEV I)
\item We will discuss computational concepts
\item Computational thinking
\item Describing computations clearly
}\end{slide}

\begin{slide}{
\item \textit{How does a programming language work?}
\item Memory, variables, conditionals, if-statements, and loops
\item These are already enough to implement anything (of course not handily!)
}\end{slide}

\SlideSubSection{At the end of the course you will be able to...}
\begin{slide}{
\item ...describe algorithms clearly
\item ...write basic programs in Python
\item ...describe the semantics of a basic Python program
}\end{slide}

\SlideSection{A programming language}
\SlideSubSection{What is programming not about?}
\begin{slide}{
\item computers
\item programming languages
\item technology
\item programs
\item websites
\item smartphones
\item ...
}\end{slide}

\SlideSubSection{What is programming about?}
\begin{slide}{
\item the encoding of logical thought
\item non-ambiguity: there is only one possible mode of execution
\item precision: there is no appeal to vagueness or intuition
}\end{slide}

\begin{slide}{
\item especially if a machine will eventually run our program
\item machines are \textbf{dumb as **ck}\footnote{rock}
}\end{slide}

\SlideSubSection{A programming language specifies}
\begin{slide}{
\item what instructions we have
\item what do they perform
\item in what order
}\end{slide}

\SlideSection{Let's start programming}
\SlideSubSection{The \texttt{stdNt} programming language}
\begin{slide}{
\item In \texttt{stdNt} we let students perform some actions
\item It does not require a machine, but only a white-board and alive (and complying students)
}\end{slide}

\begin{frame}[fragile]
\textbf{Following instructions}\footnote{The teacher should ask for a volunteer}

\begin{lstlisting}[frame=shadowbox]
take 3 steps forward
sit on the chair
turn left
slide 3 steps forward
\end{lstlisting}
\end{frame}

\SlideSubSection{The features of \texttt{stdNt} so far}
\begin{slide}{
\item \textbf{Instructions}, in English
\item \textbf{Order of execution} is left-to-right, top-to-bottom
\item \textbf{State} made up of a living, breathing student
}\end{slide}

\begin{frame}[fragile]
\textbf{Following instructions with state} \textit{(we need a ``volunteer'')}
\center
\texttt{
\begin{tabular}{| c | c | c |}
\hline
A & B & C \\
\hline
your age & 2 & -3 \\
\hline
\end{tabular}
}

\begin{lstlisting}[frame=shadowbox]
take A/4 steps forward
sit on the chair
turn left by 90 * B degrees
slide C steps forward
\end{lstlisting}
\end{frame}

\SlideSubSection{The features of \texttt{stdNt} so far}
\begin{slide}{
\item \textbf{Instructions}, in English
\item \textbf{Order of execution} is left-to-right, top-to-bottom
\item \textbf{State} made up of a living, breathing student plus a bunch of cards with data written on them
}\end{slide}

\begin{frame}[fragile]
\textbf{What if the state makes no sense?} \textit{(we need a ``volunteer'')} \\
\center
\texttt{
\begin{tabular}{| c | c | c |}
\hline
A & B & C \\
\hline
your age & ``nice day today'' & -3 \\
\hline
\end{tabular}
}

\begin{lstlisting}[frame=shadowbox]
take A/4 steps forward
sit on the chair
turn left by 90 * B degrees
slide C steps forward
\end{lstlisting}
\end{frame}

\SlideSubSection{State comes with big preconditions}
\begin{slide}{
\item It only contains information that is:
\begin{itemize}
\item used in a way that makes sense with respect to the instructions
\item logically expressed (numbers, strings, etc. rather than emotions or riddles)
\item actually accessible (there is some connection from the executor to the accessed data)
\end{itemize}
}\end{slide}

\begin{frame}[fragile]
\textbf{The state may change} \textit{(we need a ``volunteer'')} \\
\center
\texttt{
\begin{tabular}{| c | c |}
\hline
B & C \\
\hline
-1 & today's weather \\
\hline
\end{tabular}
}

\begin{lstlisting}[frame=shadowbox]
make a comment on C
write on C the index of the current day of the week
sit on the chair
turn left by 90 * B degrees
slide C steps forward
\end{lstlisting}
\end{frame}

\SlideSubSection{The features of \texttt{stdNt} so far}
\begin{slide}{
\item \textbf{Instructions}, in English
\item \textbf{Order of execution} is left-to-right, top-to-bottom
\item \textbf{\textit{Mutable} state} made up of a living, breathing student plus a bunch of cards with data written on them
}\end{slide}

\begin{frame}[fragile]
\textbf{We can make decisions}\footnote{Teacher should ask the students to perform the action} \\
\center
\texttt{
\begin{tabular}{| c | c | c | c |}
\hline
A & B & C & D \\
\hline
shirt colour & -1 & 2 & 3 \\
\hline
\end{tabular}
}

\begin{lstlisting}[frame=shadowbox]
sit on the chair
if A is ``black'' then
  turn left by 90 * B degrees
otherwise
  turn left by 90 * C degrees
clap D times
\end{lstlisting}
\end{frame}

\SlideSubSection{The features of \texttt{stdNt} so far}
\begin{slide}{
\item \textbf{Instructions}, in English
\item \textbf{Order of execution} is left-to-right, top-to-bottom
\item \textbf{Mutable state} made up of a living, breathing student plus a bunch of cards with data written on them
\item \textbf{Decisions} based on elements of the state 
}\end{slide}

\begin{frame}[fragile]
\textbf{We can repeat behavior}\footnote{The teacher should ask for two teams of volunteers} \ \\

\begin{tabular}{| l | r |}
\hline
\begin{lstlisting}[basicstyle=\ttfamily\tiny]
while there are green soldiers alive
      AND
      there are brown soldiers alive
  TEAM 1:
    a = pick green soldier
    d = pick brown soldier
    fight(a,d)
  TEAM 2:
    a = pick green soldier
    d = pick brown soldier
    fight(a,d)
\end{lstlisting}

&

\begin{lstlisting}[basicstyle=\ttfamily\tiny]
fight(a,d):
  if a = BAZOOKA AND d = GRENADIER then
    both die
  else if a = BAZOOKA then
    d dies
  else if d = GRENADIER then
    a dies
  else if brown team still has leader then
    a dies
  else
    d dies
\end{lstlisting} 

\\

\hline
\end{tabular}
\end{frame}

\SlideSubSection{The features of \texttt{stdNt} so far}
\begin{slide}{
\item \textbf{Instructions}, in English
\item \textbf{Order of execution} is left-to-right, top-to-bottom
\item \textbf{Mutable state} made up of a living, breathing student plus a bunch of cards with data written on them
\item \textbf{Decisions} based on elements of the state 
\item \textbf{Repetition} of code based on elements of the state 
}\end{slide}

\SlideSubSection{Assignment 1 in groups of four}
\begin{slide}{
\item Reprogram the game
\item Make it so that the positioning of defending soldiers makes a difference (positive or negative)
\item \textit{One group will be ``randomly selected'' to present}
}\end{slide}


\SlideSubSection{Assignment 2 in groups of four}
\begin{slide}{
\item Think about the actions needed for a game concept (at most 10).
\item Write them down and put them in the box.
\item Pick a sheet at random (if it is the one you wrote pick again).
\item Write the implementation of a game using the actions you have.
\item A group will be chosen to play the game.
}\end{slide}

\begin{frame}{This is it!}
\center
\fontsize{18pt}{7.2}\selectfont
The best of luck, and thanks for the attention!
\end{frame}

\end{document}

\begin{slide}{
\item ...
}\end{slide}

\begin{frame}[fragile]
\begin{lstlisting}
...
\end{lstlisting}
\end{frame}


