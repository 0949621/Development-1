\documentclass{beamer}
\usetheme[hideothersubsections]{HRTheme}
\usepackage{beamerthemeHRTheme}
\usepackage{graphicx}
\usepackage[space]{grffile}
\usepackage{listings}
\lstset{language=C,
basicstyle=\ttfamily\footnotesize,
mathescape=true,
breaklines=true}
\usepackage[utf8]{inputenc}
\usepackage{color}
\newcommand{\red}[1]{
\textcolor{red}{#1}
}
\newcommand{\ts}{\textbackslash}

\title{Reasoning on programs}

\author{The INFDEV Team @ HR}

\institute{Hogeschool Rotterdam \\ 
Rotterdam, Netherlands}

\date{}

\begin{document}
\maketitle

\SlideSection{Introduction}
\SlideSubSection{Lecture topics}
\begin{slide}{
\item We introduce conditional expressions
\item We show how to verify properties on complex expressions
}\end{slide}

\SlideSection{Conditional expressions}
\SlideSubSection{Conditional expressions}
\begin{slide}{
\item Sometimes we can make decisions within an expression
\item The general form is \texttt{VALUE if CONDITION else VALUE'} ($if_{VCV'}$)
\item If the condition is true, then we return \texttt{VALUE}, otherwise \texttt{VALUE'}
$$
\tiny
\left\{
\begin{matrix}
(PC,S) \overset{if_{VCV'}}{\rightarrow} V &when& (PC,S) \overset{C}{\rightarrow} TRUE & \\
(PC,S) \overset{if_{VCV'}}{\rightarrow} V' &when& (PC,S) \overset{C}{\rightarrow} FALSE & \\
\end{matrix}
\right.
$$
}\end{slide}

\begin{slide}{
\item \texttt{"adult" if age >= 18 else "minor" = ?}
\item \textbf{Can you guess the results for} \texttt{age = 18} and \texttt{age = 16}\textbf{?}
\pause
\item \texttt{age = 16}: \texttt{"minor"}
\item \texttt{age = 18}: \texttt{"adult"}
}\end{slide}

\SlideSection{Reasoning on programs}
\SlideSubSection{Reasoning on programs}
\begin{slide}{
\item Sometimes we do not know exactly the values of all variables at all times
\item The program may be too complex to allow it
}\end{slide}

\begin{frame}[fragile]
Consider a throttle control system. 

\textbf{The throttle may never go under 1000RPM}, or the engine stops and everybody dies. \\

\textbf{The temperature must be kept under control}, or the engine blows up and everybody dies. \\

\ \\
\begin{lstlisting}[frame=shadowbox,numbers=left]
throttle = throttle - 1000 if (temp > 350.0) & (throttle > 2500) else throttle
\end{lstlisting}

\ \\

The question thus is: \textbf{could the code above cause everyone to die?}

\end{frame}

\begin{frame}[fragile]
\texttt{\center
\begin{tabular}{| l | l |}
\hline
throttle & temp \\
\hline
1000..10000 & -20.0..400.0 \\
\hline
\end{tabular}
} \ \\ \ \\ \ \\

\begin{lstlisting}[frame=shadowbox,numbers=left]
throttle = throttle - 1000 if (temp > 350.0) & (throttle > 2500) else throttle
\end{lstlisting}

\pause

\ \\

\begin{tabular}{| l | l |}
\hline
throttle & temp \\
\hline
\red{?!?!?} & -20.0..400.0 \\
\hline
\end{tabular}
\end{frame}

\begin{slide}{
\item We cannot list all possible combinations of variable values
\item We cannot just ``hope it works''
\pause
\item We can reason in terms of conditions on variables
}\end{slide}

\begin{slide}{
\item We partition the state based on the conditional
\item \texttt{(temp > 350.0) \& (throttle > 2500)} generates four states
\begin{itemize}
\item \texttt{temp > 350} and \texttt{throttle > 2500}
\item \texttt{temp <= 350} and \texttt{throttle > 2500}
\item \texttt{temp > 350} and \texttt{throttle <= 2500}
\item \texttt{temp <= 350} and \texttt{throttle <= 2500}
\end{itemize}
\item We study the semantics on each of these four states
}\end{slide}

\begin{frame}[fragile]
\textbf{\texttt{temp > 350} and \texttt{throttle > 2500}}

\texttt{\center
\begin{tabular}{| l | l |}
\hline
throttle & temp \\
\hline
\red{>2500.0}..10000 & \red{>350.0}..400.0 \\
\hline
\end{tabular}
} \ \\ \ \\ \ \\

\begin{lstlisting}[frame=shadowbox,numbers=left]
throttle = throttle - 1000 if (temp > 350.0) & (throttle > 2500) else throttle
\end{lstlisting}

\pause

\texttt{\center
\begin{tabular}{| l | l |}
\hline
throttle & temp \\
\hline
\red{>1500.0..9000.0} & >350.0..400.0 \\
\hline
\end{tabular}}
\end{frame}

\begin{frame}[fragile]
\textbf{\texttt{temp <= 350} and \texttt{throttle > 2500}}

\texttt{\center
\begin{tabular}{| l | l |}
\hline
throttle & temp \\
\hline
\red{>2500.0..10000} & \red{-20.0..350.0} \\
\hline
\end{tabular}
} \ \\ \ \\ \ \\

\begin{lstlisting}[frame=shadowbox,numbers=left]
throttle = throttle - 1000 if (temp > 350.0) & (throttle > 2500) else throttle
\end{lstlisting}

\pause

\texttt{\center
\begin{tabular}{| l | l |}
\hline
throttle & temp \\
\hline
>2500.0..10000.0 & -20.0..350.0 \\
\hline
\end{tabular}}
\end{frame}

\begin{frame}[fragile]
\textbf{\texttt{temp > 350} and \texttt{throttle <= 2500}}

\texttt{\center
\begin{tabular}{| l | l |}
\hline
throttle & temp \\
\hline
\red{1000..>2500.0} & \red{>350..400.0} \\
\hline
\end{tabular}
} \ \\ \ \\ \ \\

\begin{lstlisting}[frame=shadowbox,numbers=left]
throttle = throttle - 1000 if (temp > 350.0) & (throttle > 2500) else throttle
\end{lstlisting}

\pause

\texttt{\center
\begin{tabular}{| l | l |}
\hline
throttle & temp \\
\hline
1000..>2500.0 & >350..400.0 \\
\hline
\end{tabular}}
\end{frame}


\begin{frame}[fragile]
\textbf{\texttt{temp <= 350} and \texttt{throttle <= 2500}}

\texttt{\center
\begin{tabular}{| l | l |}
\hline
throttle & temp \\
\hline
\red{1000..>2500.0} & \red{-20.0..350.0} \\
\hline
\end{tabular}
} \ \\ \ \\ \ \\

\begin{lstlisting}[frame=shadowbox,numbers=left]
throttle = throttle - 1000 if (temp > 350.0) & (throttle > 2500) else throttle
\end{lstlisting}

\pause

\texttt{\center
\begin{tabular}{| l | l |}
\hline
throttle & temp \\
\hline
\red{1000..>2500.0} & \red{-20.0..350.0} \\
\hline
\end{tabular}}
\end{frame}

\begin{slide}{
\item Each of the four states has a result
\item We now merge the results
}\end{slide}

\begin{frame}[fragile]
We now merge these states, knowing that each of them may actually happen:

\texttt{\begin{tabular}{| l | l |}
\hline
throttle & temp \\
\hline
>1500.0..9000.0 & >350.0..400.0 \\
\hline
>2500.0..10000.0 & -20.0..350.0 \\
\hline
1000.0..>2500.0 & >350..400.0 \\
\hline
1000.0..>2500.0 & -20.0..350.0 \\
\hline
\end{tabular}}

\pause

\ \\

\texttt{\begin{tabular}{| l | l |}
\hline
throttle & temp \\
\hline
\red{1000.0..10000.0} & \red{-20.0..400.0} \\
\hline
\end{tabular}}

We know that the throttle will never go below 1500RPM, and we also know that if the temperature is above 350 degrees then maximum throttle is never above 9000RPM.
\end{frame}

\begin{frame}{Conclusion?}
\center
\fontsize{18pt}{7.2}\selectfont
Nobody dies :)
\end{frame}

\begin{frame}{This is it!}
\center
\fontsize{18pt}{7.2}\selectfont
The best of luck, and thanks for the attention!
\end{frame}

\end{document}

\begin{slide}{
\item ...
}\end{slide}

\begin{frame}[fragile]
\begin{lstlisting}
...
\end{lstlisting}
\end{frame}
