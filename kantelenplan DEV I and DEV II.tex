\documentclass[12pt,a4paper,final]{article}
\usepackage[utf8]{inputenc}
\usepackage[english]{babel}
\usepackage{amsmath}
\usepackage{amsfonts}
\usepackage[table]{xcolor}
\usepackage{amssymb}
\usepackage{listings}
\usepackage{graphicx}
\usepackage{paralist}
\usepackage[left=2cm,right=2cm,top=2cm,bottom=2cm]{geometry}
\lstset{frame=lrbt}
\lstset{basicstyle=\footnotesize}

\lstset
{
	basicstyle=\scriptsize\ttfamily,
	breaklines=true,
	frame=single,
	showstringspaces=false,
	tabsize=2
}


\author{Dr. Giuseppe Maggiore, Tony Busker}
\title{Dev I and Dev II - kantelenplan}

\begin{document}
\maketitle

\section{Introduction}
In this document we give an overview of the activities of the students during the various lectures. The lectures cover a mixture of theory and practice with an interactive format:
\begin{itemize}
\item some theory will be explained in a burst of roughly half an hour; 
\item one or more exercises (in small groups) will be given; 
\item the solution to the exercises will then be presented by one of the groups
\item the lecturer and the rest of the class will assist the student called.
\end{itemize}


\section{Lectures}
In this section we describe each lecture topics and activities.


\subsection{Lecture 1 - logical model of computation}
The first lecture covers basic concepts of computation from a logical standpoint.

\paragraph*{Topics}
\begin{itemize}
\item Following a path (example: \textit{take three steps forward, turn left, ...});
\item Following a path with state (example: \textit{read N from the whiteboard, take N steps forward, ...});
\item Following a path with wrongly typed state (example: \textit{take Monday steps forward, ...});
\item Following a path with state and conditionals (example: \textit{take N steps forward if the lecturer is smiling, ....}).
\end{itemize}

\paragraph*{Activities}
\begin{itemize}
\item Let students follow instructions;
\item Introduce elements of state and let students follow instructions with state (\textit{take N/4 steps forward; N is your age});
\item Introduce elements of writable state and let students follow instructions with writable state (\textit{take N/4 steps forward; N is written under the yellow sticker});
\item Introduce elements of decision-making and let students follow instructions with state and decision making (\textit{if the sun is shining, then take N/4 steps forward; otherwise, go sit down});
\item Introduce elements of iteration and let students follow instructions with state, decision making, and iteration (\textit{divide the students in teams, and let run some script battling for the toy farm}).
\end{itemize}

\subsection{Lecture 2 - concrete model of computation}
The second lecture covers basic concepts of computation from a logical standpoint.

\paragraph*{Topics}
\begin{itemize}
\item CPU and memory;
\item a basic overview of the various things that an imperative language can do, independent of syntax;
\item introduction to semantics and post-conditions.
\end{itemize}

\paragraph*{Activities}
\begin{itemize}
\item Formalize the concept of instructions seen in the previous lecture by rewriting the scripts;
\item Formalize the concept of state and mutation seen in the previous lecture by rewriting the scripts;
\item Formalize the concept of decision-making seen in the previous lecture by rewriting the scripts;
\item Formalize the concept of iteration seen in the previous lecture by rewriting the scripts.
\end{itemize}


\subsection{Lecture 3 - Hello Python! and variables}
The third lecture covers variables and an introduction to Python.

\paragraph*{Topics}
\begin{itemize}
\item brief history of programming languages;
\item brief introduction to Python: what it does, what it does not, why we have chosen it;
\item variables in python for integers;
\item the effect of variable assignment on memory;
\end{itemize}

\subsection{Lecture 4 (practicum) - Python basics}
Set up a Python environment. Write a simple Python script that declares and assigns variables with various int, string, and float expressions.

\subsection{Lecture 5 - datatypes, and expressions}
The fourth lecture covers primitive datatypes and their associated expressions in the Python programming language.

\paragraph*{Topics}
\begin{itemize}
\item what are data-types and \textit{why we do need them}?
\item different Python data-types;
\item arithmetic expressions;
\item integer and floating point operators;
\item boolean expressions;
\item conditional expressions;
\item very long expressions with conditionals vs temporary variables: the art of naming to encode knowledge.
\end{itemize}

\paragraph*{Activities}
Call upon students to solve small riddles related to sample Python scripts on:

\begin{itemize}
\item Integers, strings, floats, bools;
\item Integer, string, float, and bool varialbes;
\item Semantics and post-conditions on variable-assignments.
\item Integers, strings, floats, bool expressions;
\item Conditional expressions;
\item Semantics and post-conditions on expressions and conditional expressions.
\end{itemize}


\subsection{Lecture 6 (practicum) - Python basics and grading}
Grading and feedback on previous practicum.


\subsection{Lecture 7 - conditional control-flow statements}
The fifth week covers conditional control-flow statements in the Python programming language.

\paragraph*{Topics}
\begin{itemize}
\item making choices;
\item \texttt{if-then} statements;
\item \texttt{if-then-else} statements;
\item the importance of an \texttt{else} statement;
\item (slightly informal) semantics;
\item exponential explosion of potential control-paths;
\item expressive power of \texttt{if-then-else}.
\end{itemize}

\paragraph*{Activities}
Call upon students to solve small riddles related to sample Python scripts on:

\begin{itemize}
\item \texttt{if-then} and \texttt{if-then-else} statements;
\item how many possible final states of a program;
\item semantics and post-conditions on conditional statements.
\end{itemize}


\subsection{Lecture 8 (practicum) - conditionals}
Write a simple \textit{turtle} script that reads some input values and makes the turtle move according to the input values by using \texttt{if} statements.


\subsection{Lecture 9 (practicum) - conditionals grading}
Grading and feedback on previous practicum.


\subsection{Lecture 10 - looping control-flow statements}
The sixth (and last) week covers looping control-flow statements in the Python programming language.

\paragraph*{Topics}
\begin{itemize}
\item repeated behaviors;
\item \texttt{while} statements;
\item (slightly informal) semantics;
\item (more than) exponential explosion of potential control-paths;
\item expressive power of \texttt{while};
\item \texttt{for} statements;
\item (slightly informal) semantics;
\item \texttt{for} as a \textit{limited} form of \texttt{while}.
\end{itemize}

\paragraph*{Activities}
Call upon students to solve small riddles related to sample Python scripts on:

\begin{itemize}
\item \texttt{while} and \texttt{for} loops;
\item how many possible final states of a program;
\item semantics and post-conditions on loops.
\end{itemize}


\subsection{Lecture 11 (practicum) - loops}
Write a simple \textit{turtle} script that makes the turtle move in a loop, square, or even spiral by using \texttt{for} or \texttt{while} statements.


\subsection{Lecture 12 (practicum) - loops grading}
Grading and feedback on previous practicum.


%\textbf{understand} & the concept of abstraction through function definition & \texttt{DEV II} & \texttt{FUNABS} %\\textbf{use and design} & functions & \texttt{DEV II} & \texttt{FUNDEF} \\
%\textbf{use and design} & recursive functions & \texttt{DEV II} & \texttt{FUNREC} \\
%\textbf{understand} & the concept of abstraction through class definition & \texttt{DEV II} & \texttt{CLSABS} \\
%\textbf{use and design} & classes without inheritance or interfaces & \texttt{DEV II} & \texttt{CLSDEF} \\
%\textbf{use} & recursively defined data structures & \texttt{DEV II} & \texttt{RECDATA} \\
%\textbf{use} & arrays & \texttt{DEV II} & \texttt{ARR} \\



\end{document}